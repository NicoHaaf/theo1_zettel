\documentclass{theozettel}

%%%%%%%%%%%%%%%%%%%%%%%%%%%%%%%%%%%%%%%%%%%%%%%%%%%%%%%%%%%%%%%%%%%%%%%%%%%%%%%%%%%%%%%%%%%%%%%%%%%%%%%%%%%%%%
% page geometry
%%%%%%%%%%%%%%%%%%%%%%%%%%%%%%%%%%%%%%%%%%%%%%%%%%%%%%%%%%%%%%%%%%%%%%%%%%%%%%%%%%%%%%%%%%%%%%%%%%%%%%%%%%%%%%
\geometry{
	left=20mm,
	right=20mm,
	top=25mm,
	bottom=20mm
}
%%%%%%%%%%%%%%%%%%%%%%%%%%%%%%%%%%%%%%%%%%%%%%%%%%%%%%%%%%%%%%%%%%%%%%%%%%%%%%%%%%%%%%%%%%%%%%%%%%%%%%%%%%%%%%

\pgfplotsset{compat=1.16}

%\renewcommand{\phi}{\varphi}

\usepackage{parskip}
\usepackage{dsfont}
\newcommand{\difd}{\text{d}}
\usepackage{titlesec} 
\titleformat{\section}[runin]
{\normalfont\large\bfseries}{\thesubsection}{1em}{}
\titleformat{\subsection}[runin]
  {\normalfont\normalsize\bfseries}{\thesubsubsection}{1em}{}
  

\theoI{5}

\begin{document}
\punkteV{5.1}{5.2}{5.3}{5.4}{5.5}

\section*{Aufgabe 7.1} 

a)

Sei $x(t)$ die Länge des Teils des Seils, das zu dem Zeitpunkt $t$ von der Tischkante hinabhängt. Die Gewichtskraft, die auf dieses Ende wirkt, ist die Kraft, die die Bewegung des Seils beeinflusst, bzw. gleich $m \ddot{x}$. Somit gilt:

$$
\ddot{x} = \frac{F_g(t)}{m} = \frac{\mu g x(t)}{m} = \frac{g}{L} x(t) =: K x(t)
$$

Um diese DGL zu lösen, machen wir den Ansatz $x(t) = e^{ct}$, setzen dies in die DGL ein und bestimmen $c$ durch umformen:

$$
c^{2} e^{ct} = K e^{ct}
$$

$$
c = \pm \sqrt{K}
$$

Somit haben wir zwei spezielle Lösungen der DGL gefunden. Die allgemeine Lösung ergibt sich nun als allgemeine Linearkombination der beiden speziellen Lösungen:

$$
x(t) = A e^{\sqrt{K} t} + B e^{- \sqrt{K} t}
$$

Wir definieren den Zeitpunkt, wann die Kette anfängt sich zu bewegen als $t_0 = 0$.

Nun können wir $A$ und $B$ mithilfe unserer Anfangsbedingungen herausfinden:

\begin{align*}
\dot{x}(0) = 0 &= \sqrt{K} (A e^{\sqrt{K} 0} - B e^{- \sqrt{K} 0}) \\
&= A - B \\
A &= B
\end{align*}


\begin{align*}
x(0) = x_0 &= A e^{\sqrt{K} 0} + B e^{- \sqrt{K} 0} \\
&= A + B \\
A &= \frac{x_0}{2} \\
B &= \frac{x_0}{2}
\end{align*}

Somit ergibt sich:

$$
x(t) = x_0 \cosh(\sqrt{K} t) = x_0 \cosh(\sqrt{\frac{g}{L}} t)
$$

Die Gleichung gilt offensichtlich nur bis zum Zeitpunkt $T$.


b)

Die Anfangsgeschwindigkeit des freien Falls ist offensichtlich $\dot{x}(T)$. Hierfür wird zunächst $T$ berechnet:

$$
x(T) =  x_0 \cosh(\sqrt{\frac{g}{L}} T) = L
$$

$$
T = \text{arccosh}(\frac{L}{g}) \sqrt{\frac{L}{g}}
$$

$\dot{x}(T)$ ergibt somit:

$$
\dot{x}(T) = x_0 \sqrt{\frac{g}{L}} \sinh(\sqrt{\frac{g}{L}} T) = x_0 \sqrt{\frac{g}{L}} \sqrt{(\cosh( \sqrt{\frac{g}{L}} \sqrt{\frac{L}{g}} \text{arccosh}(\frac{L}{x_0})) - 1)} = x_0 \sqrt{\frac{g}{x_0} - \frac{g}{L}}
$$

c) 
%Bei dieser Teilaufgabe bin ich mir nicht sicher ob ich es richtig habe (bzw auch nicht sicher, ob ich es überhaupt richtig verstanden habe.)

In diesem Fall gibt es noch die Gleitreibung, die der Gewichtskraft entgegen wirkt. Somit verändert sich die DGL wie folgt:

$$
\ddot{x} = \frac{F_g(t) - F_{Gl}(t)}{m} = \frac{\mu g x(t) - \eta (L-x(t))}{m} = (K + \frac{\eta}{m}) x - \frac{\eta L}{m}
$$

Wir definieren $K' = K + \frac{\eta}{m}$. Nun machen wir einfach den Ansatz, dass wir einen Summanden zu der Lösung der homogenen DGL $x(t)$ hinzufügen, sodass die zweite Ableitung des Summanden $- \frac{\eta L}{m}$ ergibt:

$$
x(t) = x_0 \cosh(\sqrt{K'} t) - \frac{\eta L}{2 m} t^{2}
$$

Nun müssen wir noch überprüfen, ob diese spezielle Lösung mit unseren Anfangsbedingungen übereinstimmt, und ggf. die Lösung abändern.

$$
x(0) = x_0 \cosh(0) - \frac{\eta L}{2 m} 0^{2} = x_0
$$

$$
\dot{x}(0) = x_0 \sqrt{K'} \sinh(0) - \frac{\eta L}{m} 0 = 0
$$

Somit erfüllt diese Gleichung schon tatsächlich beide unserer Anfangsbedingung, sodass $x(t) = x_0 \cosh(\sqrt{K'} t) - \frac{\eta L}{2 m} t^{2}$ tatsächlich den Abgleitvorgang in unserem Fall beschreibt.


\section*{Aufgabe 7.2} 


Einfach ausrechnen:

$$
x(0) = x_0 = a e^{- \gamma 0} + b 0 e^{- \gamma 0} = a
$$

Also gilt $a = x_0$.

$$
\dot{x}(0) = 0 = - \gamma a e^{- \gamma 0} + (b e^{- \gamma 0} - \gamma b 0 e^{- \gamma 0}) = b - \gamma a
$$

Somit gilt $b = \gamma a = \gamma x_0$.



\section*{Aufgabe 7.3} 
Bestimmen Sie die allgemeinen rellen Lösungen für die Differentialgleichungen
\subsection*{a)} $\ddot{x}-5\dot{x}+6x=0$
\begin{align*}
\ddot{x}-5\dot{x}+6x&=0\\
\Rightarrow\lambda^2-5\lambda+6&=0\\
\Leftrightarrow\left(\lambda-2\right)\cdot\left(\lambda-3\right)&=0\\
\Rightarrow\lambda_1=2 \ ;& \ \lambda_2=3\\
\Rightarrow x&=c_1e^{2t}+c_2e^{3t}
\end{align*}
\subsection*{b)}$\dddot{x}+\ddot{x}+4\dot{x}+4x=0$
\begin{align*}
\dddot{x}+\ddot{x}+4\dot{x}+4x&=0\\
\Rightarrow \lambda^3+\lambda^2+4\lambda+4&=0\\
\Leftrightarrow \lambda^2\left(\lambda+1\right)+4\left(\lambda+1\right)&=0\\
\Leftrightarrow\left(\lambda^2+4\right)\left(\lambda+1\right)&=0\\
\Rightarrow \lambda_{1;2}=0\pm 2i \ &; \ \lambda_3=-1\\
\Rightarrow x&=ae^{-x}+b\sin\left(2x\right)+c\cos\left(2x\right)
\end{align*}
\subsection*{c)}$\ddot{x}+6\dot{x}+25x=0$; $x\left(0\right)=5, \ \dot{x}\left(0\right)=-3$
\begin{align*}
\ddot{x}+6\dot{x}+25x&=0\\
\Rightarrow \lambda^2+6\lambda +25&=0\\
\Rightarrow\lambda_{1;2}=\frac{-6\pm\sqrt{6^2-4\cdot 25}}{2}&=-3\pm 4i\\
\Rightarrow x&=e^{-3t}\left(c_1\sin\left(4t\right)+c_2\cos\left(4t\right)\right)\\
\end{align*}
Anfangsbed. einsetzen:
\begin{align*}
x\left(0\right)=5&=e^{-3\cdot 0}\left(c_1\sin\left(4\cdot 0\right)+c_2\cos\left(4\cdot 0\right)\right)\\
5&=c_2\\\\
\dot{x}\left(0\right)=-3&=-3e^{-3\cdot 0}\left(c_1\sin\left(4\cdot 0\right)+5\cos\left(4\cdot 0\right)\right)+e{-3\cdot 0}\left(4c_1\cos\left(4\cdot 0\right)-20\sin\left(4\cdot 0\right)\right)\\
-3&=-15+4c_1\\
c_1&=3\\\\
\Rightarrow x&=e^{-3t}\left(3\sin\left(4t\right)+5\cos\left(4t\right)\right)
\end{align*}
\section*{Aufgabe 7.4} 



\section*{Aufgabe 7.5} 







\end{document}
