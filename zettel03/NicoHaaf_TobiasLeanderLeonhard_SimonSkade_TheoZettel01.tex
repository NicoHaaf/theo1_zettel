\documentclass{theozettel}

%%%%%%%%%%%%%%%%%%%%%%%%%%%%%%%%%%%%%%%%%%%%%%%%%%%%%%%%%%%%%%%%%%%%%%%%%%%%%%%%%%%%%%%%%%%%%%%%%%%%%%%%%%%%%%
% page geometry
%%%%%%%%%%%%%%%%%%%%%%%%%%%%%%%%%%%%%%%%%%%%%%%%%%%%%%%%%%%%%%%%%%%%%%%%%%%%%%%%%%%%%%%%%%%%%%%%%%%%%%%%%%%%%%
\geometry{
	left=20mm,
	right=20mm,
	top=25mm,
	bottom=20mm
}
%%%%%%%%%%%%%%%%%%%%%%%%%%%%%%%%%%%%%%%%%%%%%%%%%%%%%%%%%%%%%%%%%%%%%%%%%%%%%%%%%%%%%%%%%%%%%%%%%%%%%%%%%%%%%%

\pgfplotsset{compat=1.16}

\usepackage{parskip}
\usepackage{dsfont}
\newcommand{\difd}{\text{d}}
\usepackage{titlesec} 
\titleformat{\section}[runin]
{\normalfont\large\bfseries}{\thesubsection}{1em}{}
\titleformat{\subsection}[runin]
  {\normalfont\normalsize\bfseries}{\thesubsubsection}{1em}{}
\theoI{3}

\begin{document}
\punkteIV{3.1}{3.2}{3.3}{3.4}

\section*{Aufgabe 3.1} 

Zunächst wird gezeigt, dass folgendes gilt:

$$
\epsilon_{kij} \epsilon_{klm} = \delta_{il} \delta_{jm} - \delta_{im} \delta_{jl}
$$

Wenn $i = j$ ist, ist $\epsilon_{kij}$ und somit das Produkt $0$ und es gilt offensichtlich $\delta_{il} \delta_{im} - \delta_{im} \delta_{il} = 0$, sodass die Gleichung für diese Fälle stimmt.
Dies gilt analog für $l$ und $m$.

Wenn die Gleichung für den Fall $k=1$ gilt, dann gilt sei offenbar auch für die Fälle $k=2$ und $k=3$, da diese durch 'Rotation der Variablen' von $kij$ (bzw. $klm$) aus $k=1$ abgeleitet werden können und der Levi Cevita Tensor bezüglich von Rotationen der Variablen Indexe invariant ist. Somit muss nur der Fall $k=1$ betrachtet werden.

Für jedes $i,j$ mit $i \neq j$ existiert genau ein $k$, sodass der Levi Civita Tensor nicht $0$ ist (da $k$ in diesem Fall nicht $i$ oder $j$ sein darf).
Somit muss das Ergebnis für alle $i,j$ mit $i \neq j$ nur genau einmal berechnet werden, was offensichtlich auf der rechten Seite der Gleichung der Fall ist, da hier das $k$ gar nicht existiert. Wir müssen also nur zeigen, dass für das $k$, bei dem der Levi Civita Tensor nicht $0$ ist, das Ergebnis auf der rechten Seite der Gleichung rauskommt. %ich schaffe es nicht gut, das noch anschaulicher zu formulieren, aber ich denke es sollte passen
Das gleiche gilt wieder analog für $l,m$.

Der Rest kann durch ausprobieren gezeigt werden:

$$
1 = \epsilon_{123} \epsilon_{123} = \delta_{22} \delta_{33} - \delta_{23} \delta_{32} = 1 - 0 = 1
$$

$$
-1 = \epsilon_{123} \epsilon_{132} = \delta_{23} \delta_{32} - \delta_{22} \delta_{33} = 0 - 1 = -1
$$

$$
-1 = \epsilon_{132} \epsilon_{123} = \delta_{32} \delta_{23} - \delta_{33} \delta_{22} = 0 - 1 = -1
$$

$$
1 = \epsilon_{132} \epsilon_{132} = \delta_{33} \delta_{22} - \delta_{32} \delta_{23} = 1 - 0 = 1
$$

Nun soll gezeigt werden, dass für $\vec{z} = \vec{x} \cross \vec{y}$ gilt: $|z| = |x||y||sin(\theta)|$. Hierfür wird zunächst $\vec{z} \cdot \vec{z}$ betrachtet.

\begin{align}
|z|^{2} &=  \vec{z} \cdot \vec{z} \\
&= z_k z_k \\
&= (\epsilon_{kij} x_i y_j) (\epsilon_{klm} x_l y_m) \\
&= \delta_{il} \delta_{jm} x_i y_j x_l y_m - \delta_{im} \delta_{jl} x_i y_j x_l y_m \\
&= x_i^{2} y_i^{2} - (x \cdot y)^{2} \\
&= |x|^{2} |y|^{2} -  |x|^{2} |y|^{2} \cos(\theta)^{2} \\
&= |x|^{2} |y|^{2} \sin(\theta)^{2} \\
\end{align}

Da stets gilt $|z| > 0$ gilt somit:

\begin{align}
|z| &= |x| |y| |\sin(\theta)|\ \ \blacksquare
\end{align}

\newpage
\section*{Aufgabe 3.2} Bestimmen Sie die allgemeine L"osung der Differenzialgleichungen\\
\subsection*{a)} $\frac{dy}{dx} = \exp\left(-y+2x\right)$\\
\begin{align*}
					\frac{dy}{dx} 	&= e^{-y+2x}\\
									&= e^{2x}\cdot e^{-y}\\
\frac{dy}{e^{-y}}	&= e^{2x} dx\\
e^y dy  &= e^{2x} dx\\
\int{e^y dy} &= \int{e^{2x} dx}\\
e^y +c_y &= \frac{1}{2} e^{2x} + c_x = e^{2x+ \ln\left(\frac{1}{2}\right)}+c_x\\
\text{Mit }c=c_x-c_y \text{ :}\\
\Rightarrow y&= 2x+ \ln{\left(\frac{1}{2}\right)}+c
\end{align*}
\subsection*{b)}$\frac{\text{d}y}{\text{d}x} = \frac{x}{2y}\left(e^{-x}+\frac{y^2}{x^2}\right)$\\
\begin{align*}
\frac{\text{d}y}{\text{d}x}&=\frac{x}{2y}\left(e^{-x}+\frac{y^2}{x^2}\right)\\
&=\frac{x}{y}\cdot\frac{e^{-x}}{2}+\frac{y}{x}\cdot\frac{1}{2}
\end{align*}
Setze $z:=\frac{y}{x}$\\ 
\begin{align*}
&\Rightarrow y=z\cdot x\\
&\Rightarrow \frac{\text{d}y}{\text{d}x}=\frac{\text{d}z}{\text{d}x}\cdot x +z
\end{align*}
Ersetze $\frac{\text{d}y}{\text{d}x}$ mit $\frac{\text{d}z}{\text{d}x}\cdot x+z$ und $\frac{y}{x}$ mit $z$:
\begin{align*}
\frac{\text{d}z}{\text{d}x}\cdot x +z&=\frac{e^{-x}}{2z}+\frac{z}{2} &&|-\frac{z}{2}\\
\Leftrightarrow\frac{\text{d}z}{\text{d}x}\cdot x +\frac{z}{2}&=\frac{e^{-x}}{2z} &&|\cdot z\\
\Leftrightarrow\frac{\difd z}{\difd x}zx+\frac{z}{2}&=\frac{e^{-x}}{2}
\end{align*}
Beachte:
\begin{align*}
\frac{1}{2}\frac{\text{d}\left(z^2\cdot x\right)}{\text{d}x}&=\frac{1}{2}\left(2\frac{\text{d}z}{\text{d}x}zx+z\right)\\
&=\frac{\difd z}{\difd x}zx+\frac{z}{2}
\end{align*}
Ersetze $\frac{\difd z}{\difd x}zx+\frac{z}{2}$ mit $\frac{1}{2}\frac{\text{d}\left(z^2\cdot x\right)}{\text{d}x}$:
\begin{align*}
\frac{1}{2}\frac{\text{d}\left(z^2\cdot x\right)}{\text{d}x}&=\frac{e^{-x}}{z}&&|\cdot 2\\
\Leftrightarrow\frac{\difd \left(z^2\cdot x\right)}{\difd x}&=e^{-x}
\end{align*}
Dieser Term l"asst sich einfach mit d$x$ integrieren:
\begin{align*}
\int\frac{\difd \left(z^2\cdot x\right)}{\difd x}\difd x &=\int e^{-x}\difd x\\
\Leftrightarrow z^2\cdot x +c_z&=-e^{-x}+c_x
\end{align*}
Mit $c=c_x-c_z$ ergibt sich:
\begin{align*}
z^2\cdot x &=-e^{-x}+c&&|:x\\
\Leftrightarrow z^2&=\frac{-e^{-x}+c}{x}&&|\text{Resubstituiere $z$, zur Erinnerung: $z=\frac{y}{x}$}\\
\Leftrightarrow\frac{y^2}{x^2}&=\frac{-e^{-x}+c}{x}&&|\cdot x^2\\
\Leftrightarrow y^2&=\left(-e^{-x}+c\right)x &&|\sqrt{}\\
y&= \pm\sqrt{x\left(-e^{-x}+c\right)}
\end{align*}
Somit ist die allgemeine L"osung der Differenzialgleichung:
\begin{align*}
y&= \pm\sqrt{x\left(-e^{-x}+c\right)}
\end{align*}
\subsection*{c)} Bestimmen Sie die L"osung des Anfangswertproblems:
\begin{center}
$\frac{dy}{dx} = -y^2 \sin\left(x\right)$,  $ \ y\left(\frac{\pi}{2}\right)=1$
\end{center}
\begin{align*}
\frac{dy}{dx} &= -y^2 \sin\left(x\right)\\
\frac{dy}{-y^2} &= \sin\left(x\right) dx\\
\int \frac{dy}{-y^2} &= \int \sin\left(x\right) dx\\
\ y^{-1} +c_y&= -\cos\left(x\right)+c_x\\
\text{Mit }c=c_x-c_y \text{ :}\\
\ y &= -\frac{1}{\cos\left(x\right)+c}\\
\text{Anfangswert f"ur}y\left(\frac{\pi}{2}\right)=1 \text{ :}\\
1&=-\frac{1}{\cos\left(x\right)+c}\\
\Rightarrow c=-1\\
\Rightarrow y&=-\frac{1}{\cos\left(x\right)-1}\\
&=\frac{1}{1-\cos\left(x\right)}
\end{align*}
\newpage
\section*{Aufgabe 3.3}Differenzialgleichungen II\\

a)

Offensichtlich muss die DGL wie folgt aussehen:

$$
\dot{N} = - k N
$$

Hier ist es auch nicht schwer, eine Funktion zu finden, die eine Lösung dieser DGL darstellt, da der Ansatz einer Exponentialfunktion sehr vielversprechend aussieht. Man sieht such recht schnell, dass folgende Funktion die allgemeine Lösung darstellt:

$$
N(t) = c e^{-k t}
$$

Mit unserer Anfangsbedingung ergibt sich für $c$:

$$
N(0) = c = N_0
$$

Für die Halbwertszeit $\tau$ gilt:

$$
N(\tau) = \frac{N_0}{2} = N_0 e^{-k \tau}
$$

Umgeformt auf $\tau$:

$$
\tau = \frac{\ln(2)}{k}
$$

b)

In der neuen DGL gibt es noch den konstanten Summanden $\alpha$, der die Erzeugung der Atome darstellt:

$$
\dot{N} = \alpha - k N
$$

%Da $\alpha$ unabhängig von $N$ oder einer Ableitung davon ist, können $\alpha$ und $-k N$ separat betrachtet werden.
%$$
%\dot{N_1} = \alpha
%$$
%$$
%\int^{t} \dot{N_2} dt' = N(t) = \int^{t} \alpha dt' = \alpha t + C
%$$

%Die zweite Gleichung $\dot{N} = - k N$ ergibt offensichtlich das Ergebnis aus (a).

%Nun können die Gleichungen wieder zusammengefügt werden:

Dies ist der zur der DGL aus Teilaufgabe (a) korrespndierende inhomogene Fall. Deshalb wird als Ansatz die Variation der Konstanten verwendet:

$$
N(t) = C(t) e^{-kt}
$$

$$
\dot{N}(t) = \dot{C}(t) e^{-kt} - k C(t) e^{-kt}
$$

Eingesetzt in die DGL ergibt das:

$$
\dot{C}(t) e^{-kt} - k C(t) e^{-kt} + k C(t) e^{-kt} = \alpha
$$

$$
\dot{C}(t) e^{-kt} = \alpha
$$

$$
C(t) = \int^{t} \frac{\alpha}{e^{-kt'}} dt' =  \int^{t} \alpha e^{kt'} dt' = \frac{\alpha}{k} e^{kt} + C_k
$$

Die allgemeine Lösung ist somit:

$$
N(t) = (\frac{\alpha}{k} e^{kt} + C_k) e^{-kt} = \frac{\alpha}{k} + C_k e^{-kt}
$$

Lösen für die Anfangsbedingung:

$$
N(0) = N_0 = - \frac{\alpha}{k} + C_k e^{-0k} = C_k + \frac{\alpha}{k} 
$$

Somit ist $C_k = N_0 - \frac{\alpha}{k} $.

$$
N(t) = \frac{\alpha}{k} + (N_0 - \frac{\alpha}{k}) e^{-kt}
$$

Der Grenzwert für $\lim_{t \to \infty} N(t)$ ist offensichtlich $\frac{\alpha}{k}$, da $\lim_{t \to \infty} e^{-kt} = 0$.


\newpage
\section*{Aufgabe 3.4}Berechnen Sie alle ersten und zweiten partiellen Ableitungen von
\begin{center}
$F\left(x,y,z\right)=ze^{xy}+\left(x^2+y^2\right)\sin\left(z\right)+\ln\left(1+x^2+y^2\right)$
\end{center}
Zeigen Sie au"serdem f"ur dieses Beispiel, dass $\frac{\partial}{\partial x}\frac{\partial}{\partial y}F\left(x,y,z\right) =\frac{\partial}{\partial y}\frac{\partial}{\partial x}F\left(x,y,z\right)$ (Satz von Schwarz) und genauso f"ur $x$,$z$ und $y$,$z$.\\\\
Zuerst die erste und zweite partielle Ableitung:\\\\
F"ur $x$:\\
\begin{align*}
\frac{\partial}{\partial x}F\left(x,y,z\right) &=\frac{\partial}{\partial x}ze^{xy}+\left(x^2+y^2\right)\sin\left(z\right)+\ln\left(1+x^2+y^2\right)\\
&=yze^{xy}+2x\sin\left(z\right)+\frac{2x}{1+x^2+y^2}\\\\
\frac{\partial^2}{\partial x^2}F\left(x,y,z\right) &=\frac{\partial^2}{\partial x^2}ze^{xy}+\left(x^2+y^2\right)\sin\left(z\right)+\ln\left(1+x^2+y^2\right)\\
&=\frac{\partial}{\partial x}yze^{xy}+2x\sin\left(z\right)+\frac{2x}{1+x^2+y^2}\\
&= y^2 z e^{xy}+2\sin\left(z\right) +2 \cdot\frac{1-x^2+y^2}{\left(1+x^2+y^2\right)^2}
\end{align*}

F"ur $y$:\\
\begin{align*}
\frac{\partial}{\partial y}F\left(x,y,z\right) &=\frac{\partial}{\partial y}ze^{xy}+\left(x^2+y^2\right)\sin\left(z\right)+\ln\left(1+x^2+y^2\right)\\
&=xze^{xy}+2y\sin\left(z\right)+\frac{2y}{1+x^2+y^2}\\\\
\frac{\partial^2}{\partial y^2}F\left(x,y,z\right) &=\frac{\partial^2}{\partial y^2}ze^{xy}+\left(x^2+y^2\right)\sin\left(z\right)+\ln\left(1+x^2+y^2\right)\\
&=\frac{\partial}{\partial y}xze^{xy}+2y\sin\left(z\right)+\frac{2y}{1+x^2+y^2}\\
&= x^2 z e^{xy}+2\sin\left(z\right) +2 \cdot\frac{1+x^2-y^2}{\left(1+x^2+y^2\right)^2}
\end{align*}

F"ur $z$:\\
\begin{align*}
\frac{\partial}{\partial z}F\left(x,y,z\right) &=\frac{\partial}{\partial z}ze^{xy}+\left(x^2+y^2\right)\sin\left(z\right)+\ln\left(1+x^2+y^2\right)\\
&=e^{xy}+\cos\left(z\right)\cdot\left(x^2+y^2\right)\\\\
\frac{\partial^2}{\partial z^2}F\left(x,y,z\right) &=\frac{\partial^2}{\partial z^2}ze^{xy}+\left(x^2+y^2\right)\sin\left(z\right)+\ln\left(1+x^2+y^2\right)\\
&=\frac{\partial}{\partial z}e^{xy}+\left(x^2+y^2\right)\cos\left(z\right)\\
&= -\sin\left(z\right) \cdot \left(x^2+y^2\right)
\end{align*}
Nun der Satz von Schwarz:\\\\
F"ur $\frac{\partial}{\partial x}\frac{\partial}{\partial y}F\left(x,y,z\right) =\frac{\partial}{\partial y}\frac{\partial}{\partial x}F\left(x,y,z\right)$ :
\begin{align*}
\frac{\partial}{\partial x}\frac{\partial}{\partial y}F\left(x,y,z\right) &= \frac{\partial}{\partial x}\frac{\partial}{\partial y}ze^{xy}+\left(x^2+y^2\right)\sin\left(z\right)+\ln\left(1+x^2+y^2\right)\\
&=\frac{\partial}{\partial x}xze^{xy}+2y\sin\left(z\right)+\frac{2y}{1+x^2+y^2}\\
&= ze^{xy}+xze^{xy}-\frac{4xy}{\left(1+x^2+y^2\right)^2}\\\\
\frac{\partial}{\partial y}\frac{\partial}{\partial x}F\left(x,y,z\right) &= \frac{\partial}{\partial y}\frac{\partial}{\partial x}ze^{xy}+\left(x^2+y^2\right)\sin\left(z\right)+\ln\left(1+x^2+y^2\right)\\
&= \frac{\partial}{\partial y}yze^{xy}+2x\sin\left(z\right)+\frac{2x}{1+x^2+y^2}\\
&= ze^{xy}+xze^{xy}-\frac{4xy}{\left(1+x^2+y^2\right)^2}\\\\
\Rightarrow \frac{\partial}{\partial x}\frac{\partial}{\partial y}F\left(x,y,z\right) &=\frac{\partial}{\partial y}\frac{\partial}{\partial x}F\left(x,y,z\right)\\
\end{align*}
F"ur $\frac{\partial}{\partial y}\frac{\partial}{\partial z}F\left(x,y,z\right) =\frac{\partial}{\partial z}\frac{\partial}{\partial y}F\left(x,y,z\right)$ :
\begin{align*}
\frac{\partial}{\partial y}\frac{\partial}{\partial z}F\left(x,y,z\right) &= \frac{\partial}{\partial y}\frac{\partial}{\partial z}ze^{xy}+\left(x^2+y^2\right)\sin\left(z\right)+\ln\left(1+x^2+y^2\right)\\
&=\frac{\partial}{\partial y}e^{xy}+\cos\left(z\right)\cdot\left( x^2+y^2\right)\\
&= xe^{xy}+2y\cos\left(z\right)\\\\
\frac{\partial}{\partial z}\frac{\partial}{\partial y}F\left(x,y,z\right)&= \frac{\partial}{\partial z}\frac{\partial}{\partial y}ze^{xy}+\left(x^2+y^2\right)\sin\left(z\right)+\ln\left(1+x^2+y^2\right)\\
&= \frac{\partial}{\partial z}xze^{xy}+2y\sin\left( z\right)+\frac{2y}{1+x^2+y^2}\\
&= xe^{xy}+2y\cos\left(z\right)\\\\
\Rightarrow \frac{\partial}{\partial y}\frac{\partial}{\partial z}F\left(x,y,z\right) &=\frac{\partial}{\partial z}\frac{\partial}{\partial y}F\left(x,y,z\right)\\
\end{align*}
F"ur $\frac{\partial}{\partial z}\frac{\partial}{\partial x}F\left(x,y,z\right) =\frac{\partial}{\partial x}\frac{\partial}{\partial z}F\left(x,y,z\right)$ :
\begin{align*}
\frac{\partial}{\partial z}\frac{\partial}{\partial x}F\left(x,y,z\right) &= \frac{\partial}{\partial z}\frac{\partial}{\partial x}ze^{xy}+\left(x^2+y^2\right)\sin\left(z\right)+\ln\left(1+x^2+y^2\right)\\
&=\frac{\partial}{\partial z}yze^{xy}+2x\sin\left(z\right)+\frac{2x}{1+x^2+y^2}\\
&= ye^{xy}+2x\cos\left(z\right)\\\\
\frac{\partial}{\partial x}\frac{\partial}{\partial z}F\left(x,y,z\right)&= \frac{\partial}{\partial x}\frac{\partial}{\partial z}ze^{xy}+\left(x^2+y^2\right)\sin\left(z\right)+\ln\left(1+x^2+y^2\right)\\
&= \frac{\partial}{\partial x}e^{xy}+\cos\left(z\right)\cdot\left(x^2+y^2\right)\\
&= ye^{xy}+2x\cos\left(z\right)\\\\
\Rightarrow \frac{\partial}{\partial z}\frac{\partial}{\partial x}F\left(x,y,z\right) &=\frac{\partial}{\partial x}\frac{\partial}{\partial z}F\left(x,y,z\right)
\end{align*}
\newpage
\section*{Aufgabe 3.5} Trajektorie und co.








\end{document}
