\documentclass{theozettel}

%%%%%%%%%%%%%%%%%%%%%%%%%%%%%%%%%%%%%%%%%%%%%%%%%%%%%%%%%%%%%%%%%%%%%%%%%%%%%%%%%%%%%%%%%%%%%%%%%%%%%%%%%%%%%%
% page geometry
%%%%%%%%%%%%%%%%%%%%%%%%%%%%%%%%%%%%%%%%%%%%%%%%%%%%%%%%%%%%%%%%%%%%%%%%%%%%%%%%%%%%%%%%%%%%%%%%%%%%%%%%%%%%%%
\geometry{
	left=20mm,
	right=20mm,
	top=25mm,
	bottom=20mm
}
%%%%%%%%%%%%%%%%%%%%%%%%%%%%%%%%%%%%%%%%%%%%%%%%%%%%%%%%%%%%%%%%%%%%%%%%%%%%%%%%%%%%%%%%%%%%%%%%%%%%%%%%%%%%%%

\pgfplotsset{compat=1.16}

\usepackage{dsfont}

\theoI{5}

\begin{document}
\punkteV{4.1}{4.2}{4.3}{4.4}{4.5}

\section*{Aufgabe 5.1}




\section*{Aufgabe 5.2}




\section*{Aufgabe 5.3}




\section*{Aufgabe 5.4}

Sei $\Sigma$ die vom vorgegebenen Rechteck eingeschlossene Fläche und $R$ der Rand dieser Fläche, über den beim Kurvenintegral integriert wird.

Zunächst wird das Kurvenintegral evaluiert. Da wir über das vorgegebene Rechteck integrieren, bei dessen Kanten sich jeweils nur eine Komponente ändert, gilt:

\begin{align}
\oint_R \vec{F} \cdot d \vec{s} &= \int_0^{1} F_x(s,0,0) ds + \int_0^{1} F_y(1,s,0) ds + \int_1^{0} F_x(s,1,0) ds + \int_1^{0} F_y(0,s,0) ds \\
&= \int_0^{1} F_x(s,0,0) - F_x(s,1,0) ds + \int_0^{1} F_y(1,s,0) - F_y(0,s,0) ds \\
&=  \int_0^{1} 2s^{2} - (2s^{2} - 3) ds + \int_0^{1} 0-0 ds \\
&= 3
\end{align}

Nach dem Satz von Stokes sollte dies gleich dem folgenden Flächenintegral sein:

$$
\int_\Sigma (\vec{\nabla} \times \vec{F}) \cdot d \vec{f}
$$

Da die Fläche in der $xy$-Ebene liegt, interessiert uns nur der Anteil von $(\vec{\nabla} \times \vec{F})$, der senkrecht dazu, also in $z$-Richtung ausgerichtet ist. Dieser ist:

$$
(\vec{\nabla} \times \vec{F})_z = \frac{\partial}{\partial x} 4yz - \frac{\partial}{\partial y} (2x^{2} - 3y) = 3
$$

Die Integration über die infinitesimalen Flächenelemente kann nun so aufgelößt werden, dass man zunächst über $x$ und dann über $y$ integriert:

$$
\int_\Sigma (\vec{\nabla} \times \vec{F}) \cdot d \vec{f} = \int_0^{1}  \int_0^{1} 3\ dx\ dy = 3
$$

Somit stimmt das Ergebnis überein und das Beispiel widerlegt den Satz von Stokes nicht.


\end{document}
