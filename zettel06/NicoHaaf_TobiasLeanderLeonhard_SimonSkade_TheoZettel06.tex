\documentclass{theozettel}

%%%%%%%%%%%%%%%%%%%%%%%%%%%%%%%%%%%%%%%%%%%%%%%%%%%%%%%%%%%%%%%%%%%%%%%%%%%%%%%%%%%%%%%%%%%%%%%%%%%%%%%%%%%%%%
% page geometry
%%%%%%%%%%%%%%%%%%%%%%%%%%%%%%%%%%%%%%%%%%%%%%%%%%%%%%%%%%%%%%%%%%%%%%%%%%%%%%%%%%%%%%%%%%%%%%%%%%%%%%%%%%%%%%
\geometry{
	left=20mm,
	right=20mm,
	top=25mm,
	bottom=20mm
}
%%%%%%%%%%%%%%%%%%%%%%%%%%%%%%%%%%%%%%%%%%%%%%%%%%%%%%%%%%%%%%%%%%%%%%%%%%%%%%%%%%%%%%%%%%%%%%%%%%%%%%%%%%%%%%

\pgfplotsset{compat=1.16}

%\renewcommand{\phi}{\varphi}

\usepackage{parskip}
\usepackage{dsfont}
\newcommand{\difd}{\text{d}}
\usepackage{titlesec} 
\titleformat{\section}[runin]
{\normalfont\large\bfseries}{\thesubsection}{1em}{}
\titleformat{\subsection}[runin]
  {\normalfont\normalsize\bfseries}{\thesubsubsection}{1em}{}
  

\theoI{5}

\begin{document}
\punkteV{5.1}{5.2}{5.3}{5.4}{5.5}

\section*{Aufgabe 6.1} 
Betrachten Sie das eindimensionale Potential
\begin{align*}
V\left(x\right)=-\frac{\alpha}{x}+\frac{\beta}{x^2}, \ \alpha,\beta>0
\end{align*}
für $x>0$.
\subsection*{a)}
Bestimmen Sie das Minimum des Potentials.\\\\
Wir bestimmen die Nullstelle(n) der ersten Ableitung des Potentials:
\begin{align*}
\frac{\text{d}V\left(x\right)}{\text{d}x}=\frac{\text{d}}{\text{d}x}\left(-\frac{\alpha}{x}+\frac{\beta}{x^2}\right)&=\frac{\alpha}{x^2}-2\frac{\beta}{x^3}\\
0&=\frac{\alpha}{x^2}-2\frac{\beta}{x^3}\\
\Rightarrow x&=2\frac{\beta}{\alpha}
\end{align*}
Da dies auch ein Maximum des Potentials sein könnte, bestimmen wir noch die zweite Ableitung:
\begin{align*}
\frac{\text{d}}{\text{d}x}\left(\frac{\alpha}{x^2}-2\frac{\beta}{x^3}\right)&=-2\frac{\alpha}{x^3}+6\frac{\beta}{x^4}\\
\text{Setze} \ x=2\frac{\beta}{\alpha}&\\
&=2\frac{\alpha}{\left(2\frac{\beta}{\alpha}\right)^3}
+6\frac{\beta}{\left(2\frac{\beta}{\alpha}\right)^4}\\
&=3\frac{\alpha^4}{8\beta^3}-2\frac{\alpha^4}{8\beta^3}\\
&=\frac{\alpha^4}{8\beta^3}>0
\end{align*}
Da diese an der Stelle $2\frac{\beta}{\alpha}$ größer als $0$ ist, handelt es sich um ein Minimum.\\
Somit liegt das Minimum des Potentials bei $2\frac{\beta}{\alpha}$.
\subsection*{b)}
Berechnen Sie die Taylor-Entwicklung des Potentials bis einschließlich der zweiten Ordnung um das Minimum.\\\\
Wir setzen einfach die Ableitungen aus \textbf{a)} in die Formel für die Taylor-Reihe ein, dabei beachten wir, das $V^{\left(1\right)}\left(2\frac{\beta}{\alpha}\right)=0$ und dieser Teil schon wegfällt:
\begin{align*}
V\left(x\right)\big|_{2\frac{\beta}{\alpha}}&=-\frac{\alpha^2}{4\beta}+0+\frac{1}{2}\cdot\frac{\alpha^4}{8\beta^3}\cdot\left(x-2\frac{\beta}{\alpha}\right)^2\\
&=-\frac{\alpha^2}{4\beta}+\frac{\alpha^4}{16\beta^3}x^2-\frac{\alpha^3}{4\beta^2}x+\frac{\alpha^2}{4\beta}\\
&=\frac{\alpha^4}{16\beta^3}x^2-\frac{\alpha^3}{4\beta^2}x
\end{align*}
\subsection*{c)}
Berechnen Sie die Kreisfrequenz $\omega$ der Schwingung eines Teilchens der Masse $m$ bei kleinen Auslenkungen um seine Ruhelage im Minimum\\\\
Für kleine Auslenkungen können wir die Taylor-Reihe aus \textbf{b)} nutzen, es gilt:
\begin{align*}
\text{F}\left(x\right)=m\ddot{x}=-kx&=-\frac{\text{d}V\left(x\right)}{\text{d}x}=-\frac{\text{d}}{\text{d}x}\left(\frac{\alpha^4}{16\beta^3}x^2-\frac{\alpha^3}{4\beta^2}x\right)\\
m\cdot\ddot{x}&=-\frac{\alpha^4}{8\beta^3}x+\frac{\alpha^3}{4\beta^2}\\
\ddot{x}&=-\frac{\alpha^4}{8m\beta^3}x+\frac{\alpha^3}{4\beta^2}\\\\
\text{Eine mögliche Lösung dieser Dgl. ist:}\\
x\left(t\right)&=\sin\left(\sqrt{\frac{\alpha^4}{8m\beta^3}} \ t\right)+\frac{\alpha^3}{8\beta^2}t^2\\
\Rightarrow \omega &=\sqrt{\frac{\alpha^4}{8m\beta^3}}
\end{align*}


\newpage
\section*{Aufgabe 6.2} 

\begin{center}
\includegraphics[width=15cm]{A2-Teil1.pdf}\\
\includegraphics[width=15cm]{A2-Teil2_annotated.pdf}\\
\includegraphics[width=15cm]{A2-Teil3_annotated.pdf}
\end{center}

* Ergänzung 1:

Wegen der Argumentation am Anfang vom 3. eingescannten Blatt gilt:

$$
\arcsin(\sqrt{\frac{k}{2 E'}} x_0) = \pm \frac{\pi}{2}
$$

Da gilt $\sin(x \pm \frac{\pi}{2}) = \pm \cos(x)$, lässt sich die Gleichung für $x$ vereinfachen zu (es wurde $E'$ resubstituiert):

$$
x(t) = \sqrt{\frac{2 E + k x_0^{2}}{k}} \cos(\sqrt{\frac{k}{m}} t)
$$

Dies ist auch in der ersten Skizze dargestellt.


** Ergänzung 2:

Bitte nur die schwarze, nicht die mit Bleistift gezeichnete Linie betrachten. Die Gleichung geht wie folgt weiter:

\begin{align}
\frac{k}{2}x^{2} - \frac{k}{2}x_0^{2} &= \frac{k}{2}\left(\sqrt{\frac{2 E + k x_0^{2}}{k}} \cos(\sqrt{\frac{k}{m}} t)\right)^{2} - \frac{k}{2}x_0^{2} \\
&= E \cos(\sqrt{\frac{k}{m}} t)^{2} + \frac{k}{2}x_0^{2} (\cos(\sqrt{\frac{k}{m}} t) - 1)^{2} \\
&= E \cos(\sqrt{\frac{k}{m}} t)^{2} - \frac{k}{2}x_0^{2} \sin(\sqrt{\frac{k}{m}} t)^{2}\\
&=  E \left( \cos(\sqrt{\frac{k}{m}} t)^{2} - \sin(\sqrt{\frac{k}{m}} t)^{2} \right)
\end{align}

Der letzte Umformungsschritt geht ebenfalls aus der Tatsache hervor, dass $\dot{x}(0) = 0$, sodass zum Zeitpunkt $0$ die gesamte Energie in der potentiellen Energie stecken muss.


\newpage
\section*{Aufgabe 6.3} 
Gegeben sei das Vektorfeld:
\[
	\vec{F}(\vec{x}) = \mqty(x \\ x^{2} + y^{2} + z^{2} \\ ye^{z}).
\]
Die gemeinsamen Punkte der Ebene $z=y$ und des Kreiszylinders, dessen Querschnittsfläche den Radius $1$ besitzt und dessen Achse mit der $z$-Achse zusammenfällt, definieren eine Schnittkurve $S$. Der Weg $C$ beginne im Punkt $\mqty(1, 0, 0)$, verlaufe entlang der Schnittkurve $S$ im Halbraum $y \geq 0$ und ende im Punkte $\mqty(0, 1, 1)$.

\subsection*{(a)}Skizze des Kreiszylinders, der Ebene und des Wegs $C$:
	\begin{figure}[H]
	\centering
	\includegraphics[scale=1]{zylinderplot.pdf}
	\end{figure}
Wir bestimme explizit die Parametrisierung $\vec{x}(\phi)$ der Schnittkurve $S$: wir setzen in die allgemeine Parametrisierung des Kreiszylinders ein, dass gilt $z(\phi,z) = y(\phi,z)$
	\[
		\vec{x}(\phi) = \vec{x}(\phi,z) = \mqty(\cos(\phi) \\ \sin(\phi) \\ \sin(\phi)).
	\]
Für $\phi \in \qty[0,\frac{\pi}{2}]$ beschreibt diese Parametrisierung genau die Schnittkurve $S$, denn $\vec{x}(0) = \mqty(1,0,0)$ und $\vec{x}(\frac{\pi}{2}) = \mqty(0,1,1)$, nach Konstruktion liegt $\vec{x}(\phi)$ auf dem Kreiszylinder und wegen $z(\phi) = \sin(\phi) = y(\phi)$  liegt $\vec{x}(\phi)$ in der Ebene mit $z=y$. Außerdem $y(\phi) \neq 1$ für alle $\phi \in [0,\frac{\phi}{2})$ und $y(\phi) = \sin(\phi) \geq 0$ für alle $\phi \in \qty[0,\frac{\pi}{2}]$. Somit beschreibt $\vec{x} \colon \qty[0,\frac{\pi}{2}] \to \R^{3}, \phi \mapsto \vec{x}(\phi)$ genau die Schnittkurve $S$. 

\subsection*{(b)} 
Es gilt:
	\[
		\dv{\vec{x}(\phi)}{\phi} = \mqty(-\sin(\phi) \\ \cos(\phi) \\ \cos(\phi) ).
	\]
Wir berechnen: bei (1) wird durch $u = \sin(\phi)$ substituiert, bei (2) wird benutzt, dass $(xe^{x}-e^{x})' = xe^{x}$. 
	\begin{align*}
		\int_{C} \vec{F}(\vec{x}) \cdot \dd \vec{x} &= \int_{0}^{\frac{\pi}{2}} \vec{F}(\vec{x}(\phi)) \cdot \dv{\vec{x}(\phi)}{\phi} \dd \phi \\
		&= \int_{0}^{\frac{\pi}{2}} \mqty(\cos(\phi) \\  \cos^{2}(\phi) + 2\sin^{2}(\phi) \\ \sin(\phi)\exp(\sin(\phi)) ) \cdot \mqty(-\sin(\phi) \\ \cos(\phi) \\ \cos(\phi) ) \dd \phi \\
		&= \int_{0}^{\frac{\pi}{2}} -\cos(\phi)\sin(\phi) + [1 + \sin^{2}(\phi)]\cos(\phi) + \cos(\phi)\sin(\phi)\exp(\sin(\phi)) \dd \phi \\
		&= \int_{0}^{\frac{\pi}{2}} \cos(\phi)\qty[-\sin(\phi) + 1 + \sin^{2}(\phi) + \sin(\phi)\exp(\sin(\phi))] \dd \phi \\
		&\stackrel{\text{(1)}}{=} \int_{0}^{1} \qty[1 - u +u^{2} + ue^{u}] \cos(\phi) \frac{\dd u}{\cos(\phi)} \\
		&= \int_{0}^{1} 1-u+u^{2}+ue^{u} \dd u \\
		&\stackrel{\text{(2)}}{=} \eval{u - \frac{1}{2}u^{2} + \frac{1}{3}u^{3} + ue^{u} - e^{u}}_{0}^{1} \\
		&= 1 - \frac{1}{2} + \frac{1}{3} + e - e - 0 + 0 - 0 + 0 + 1  = \frac{11}{6}.
	\end{align*} \hfill $\square$




\newpage
\section*{Aufgabe 6.4} 
Berechnen Sie
\subsection*{a)}die allgemeine Lösung der Differentialgleichung
\begin{align*}
y'=-\frac{2}{x}y+\frac{\ln x}{x^2}\tag*{(1)}\\
\end{align*}
Wir lösen zuerst den homogenen Teil:
\begin{align*}
\frac{\text{d}y}{\text{d}x}&=-\frac{2}{x}y\\
\text{Sep. der Var.:}\\
\frac{\text{d}y}{y}&=-\frac{2}{x}\text{d}x&&|\int\\
\int\frac{\text{d}y}{y}&=\int-\frac{2}{x}\text{d}x\\
\ln\left(y\right)&=-2\left(\ln\left(x\right)+\underline{c}\right)\\
\ln\left(y\right)&=-2\ln\left(x\right)-2\underline{c}&&|\text{$\ln$ auflösen}\\
y&=e^{-2\ln\left(x\right)}\cdot e^{-2\underline{c}}&&|c:=e^{-2\underline{c}}\\
y&=cx^{-2}\tag*{(2)}\\\\
\text{Variation der Kons.:} \ c&\rightarrow c\left(x\right)\\
y'&=-2\frac{c\left(x\right)}{x^2}+c'\left(x\right)x^{-2}\tag*{(3)}\\
\text{Setze (2) und (3) in (1) ein:}\\
-2\frac{c\left(x\right)}{x^3}+\frac{c'\left(x\right)}{x^2}&=-\frac{2}{x}\cdot\frac{c\left(x\right)}{x^2}+\frac{\ln\left(x\right)}{x^2}\\
\Leftrightarrow \frac{c'\left(x\right)}{x^2}&=\frac{\ln\left(x\right)}{x^2}\\
\Leftrightarrow c'\left(x\right)&=\ln\left(x\right)\\
\Rightarrow c\left(x\right)&=\int\text{d}x \ c'\left(x\right) =\int\text{d}x \ \ln\left(x\right)=x\ln\left(x\right)-x+\hat{c} \tag*{(4)}\\\\
\text{(4) in (2):}\\
y&=\left(x\ln\left(x\right)-x+\hat{c}\right)\cdot \frac{1}{x^2}\\
y&=\frac{\ln\left(x\right)-1}{x}-\frac{\hat{c}}{x^2}
\end{align*}
\newpage
\subsection*{b)} die Lösung des Anfangswertproblems
\begin{align*}
y'=-2xy+2xe^{-x^2} \tag*{(1)}, \ y\left(0\right)=2
\end{align*}
Wir bestimmen den homogenen Teil:
\begin{align*}
\frac{\text{d}y}{\text{d}x}&=-2xy\\
\text{Sep. der Var.:}\\
\frac{\text{d}y}{y}=-2x\text{d}x&&|\int\\
\int\text{d}y \ \frac{1}{y}&=-2\int\text{d}x \ x\\
\ln\left(y\right)&=-x^2-\hat{c}&&|\text{$\ln$ beseitigen}\\
y&=e^{-x^2-\hat{c}}\\
y&=e^{-x^2}\cdot e^{-\hat{c}}&&|c:=e^{-\hat{c}}\\
y&=ce^{-x^2}\tag*{(2)}\\\\
\text{Variation der Kons.:} \ c\rightarrow c\left(x\right)\\
y'&=c'\left(x\right)e^{-x^2}-2xe^{-x^2}c\left(x\right)\tag*{(3)}\\\\
\text{(2) und (3) in (1):}\\
c'\left(x\right)e^{-x^2}-2xe^{-x^2}&=-2xc\left(x\right)e^{-x^2}+2xe^{-x^2}\\
c'\left(x\right)e^{-x^2}&=2xe^{-x^2}\\
c'\left(x\right)&=2x\\
\Rightarrow c\left(x\right)&=\int\text{d}x \ c'\left(x\right)=\int\text{d}x \ 2x=x^2+\hat{c}\tag*{(4)}\\\\
\text{(4) in (1) ergibt allg. Lösung für Dgl.:}\\
y&=\left(x^2+\hat{c}\right)e^{-x^2}\\\\
\text{Anfangsbed.} \ y\left(0\right)=2 \ \text{einsetzen:}\\
y\left(0\right)&=\hat{c}=2\\
\Rightarrow y&=\left(x^2+2\right)e^{-x^2}
\end{align*}



\end{document}
