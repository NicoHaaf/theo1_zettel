\documentclass{theozettel}

%%%%%%%%%%%%%%%%%%%%%%%%%%%%%%%%%%%%%%%%%%%%%%%%%%%%%%%%%%%%%%%%%%%%%%%%%%%%%%%%%%%%%%%%%%%%%%%%%%%%%%%%%%%%%%
% page geometry
%%%%%%%%%%%%%%%%%%%%%%%%%%%%%%%%%%%%%%%%%%%%%%%%%%%%%%%%%%%%%%%%%%%%%%%%%%%%%%%%%%%%%%%%%%%%%%%%%%%%%%%%%%%%%%
\geometry{
	left=20mm,
	right=20mm,
	top=25mm,
	bottom=20mm
}
%%%%%%%%%%%%%%%%%%%%%%%%%%%%%%%%%%%%%%%%%%%%%%%%%%%%%%%%%%%%%%%%%%%%%%%%%%%%%%%%%%%%%%%%%%%%%%%%%%%%%%%%%%%%%%

\pgfplotsset{compat=1.16}

\usepackage{dsfont}

\theoI{5}

\begin{document}
\punkteV{5.1}{5.2}{5.3}{5.4}{5.5}

\section*{Aufgabe 5.1}


\newpage
\section*{Aufgabe 5.2}


\newpage
\section*{Aufgabe 5.3}


\newpage
\section*{Aufgabe 5.4}
Wir bestimmen eine Stammfunktion von 
	\[
		f(x) = \frac{1}{a+bx+cx^2}, \ \ \ \ \ b^{2} > 4ac.
	\]
Für die Nullstellen von $a+bx+cx^{2}$ gilt:
	\[
		x_{1} = \frac{-b + \sqrt{b^{2}-4ac}}{2c}, \ \ \ \ \ x_{2} = \frac{-b - \sqrt{b^{2}-4ac}}{2c}.
	\]
Wir wählen den Ansatz wie gegeben:
	\[
		\frac{1}{a+bx+cx^2} = \frac{\alpha}{x-x_{1}} + \frac{\beta}{x-x_{2}}
	\]
und erhalten:
	\[
		0 \cdot x + 1 = (\alpha + \beta) \cdot x + (- \alpha x_{2} - \beta x_{1})
	\]
Koeffizientenvergleich liefert:
	\begin{align*}
		0 &= \alpha + \beta, \\
		1 &= -\alpha x_{2} - \beta x_{1}.
	\end{align*}
Woraus wir erhalten $\alpha = - \beta$ und somit:
	\[
		\alpha (x_{1} - x_{2}) = 1, \ \ \ \implies \alpha = \frac{1}{x_{1} - x_{2}} = \frac{c}{\sqrt{b^{2}-4ac}}.
	\]
Somit erhalten wir eine Stammfunktion von $f$ durch:
	\begin{align*}
		\int f \dd x &= \int \frac{1}{a+bx+cx^2} \dd x \\
		&= \int \frac{\alpha}{x-x_{1}} + \frac{\beta}{x-x_{2}} \dd x \\
		&= \alpha \ln(x-x_{1}) + \beta \ln(x-x_{2}) \\
		&= \alpha \qty(\ln(x-x_{1}) - \ln(x-x_{2})) \\
		&= \frac{c}{\sqrt{b^{2}-4ac}} \qty[ \ln(x - \frac{-b + \sqrt{b^{2}-4ac}}{2c}) - \ln(x + \frac{+b + \sqrt{b^{2}-4ac}}{2c}) ]
	\end{align*}


\newpage
\section*{Aufgabe 5.5}









\end{document}