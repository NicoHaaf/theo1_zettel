\documentclass{theozettel}

%%%%%%%%%%%%%%%%%%%%%%%%%%%%%%%%%%%%%%%%%%%%%%%%%%%%%%%%%%%%%%%%%%%%%%%%%%%%%%%%%%%%%%%%%%%%%%%%%%%%%%%%%%%%%%
% page geometry
%%%%%%%%%%%%%%%%%%%%%%%%%%%%%%%%%%%%%%%%%%%%%%%%%%%%%%%%%%%%%%%%%%%%%%%%%%%%%%%%%%%%%%%%%%%%%%%%%%%%%%%%%%%%%%
\geometry{
	left=20mm,
	right=20mm,
	top=25mm,
	bottom=20mm
}
%%%%%%%%%%%%%%%%%%%%%%%%%%%%%%%%%%%%%%%%%%%%%%%%%%%%%%%%%%%%%%%%%%%%%%%%%%%%%%%%%%%%%%%%%%%%%%%%%%%%%%%%%%%%%%

\pgfplotsset{compat=1.16}

\usepackage{dsfont}

\theoI{4}

\begin{document}
\punkteV{4.1}{4.2}{4.3}{4.4}{4.5}

\section*{Aufgabe 4.1}Sei $\vec{r}=\left(x,y,z\right)^T$, $r=\left|\vec{r}\right|$, $\hat{\vec{r}}=\frac{\vec{r}}{r}$ und $\vec{\nabla}=\left(\partial_x,\partial_y,\partial_z\right)^T$. Zeigen Sie, dass
\subsection*{a)}$\vec{\nabla}r=\hat{\vec{r}}$
\begin{align*}
\vec{\nabla}r&=\hat{\vec{r}}\\
\mqty(\partial_x\\\partial_y \\\partial_x  )\cdot r&=\mqty(\partial_x\cdot r\\\partial_y\cdot r \\\partial_x\cdot r  )\\
&=\mqty(\frac{1}{2}\cdot r^{-1}\cdot 2x\\\frac{1}{2}\cdot r^{-1}\cdot 2y\\\frac{1}{2}\cdot r^{-1}\cdot 2z)\\
&=\mqty(xr^{-1}\\yr^{-1}\\zr^{-1})\\
&=\mqty(x\\y\\z)\cdot r^{-1}\\
&=\frac{\vec{r}}{r}\\
&=\hat{\vec{r}}
\end{align*}

\subsection*{b)}$\vec{\nabla}f\left(r\right)=\hat{\vec{r}} \ \frac{\text{d}f}{dr}$
\begin{align*}
\vec{\nabla}f\left(r\right)&=\hat{\vec{r}} \ \frac{\text{d}f}{dr}\\
\mqty(\partial_x\\\partial_y \\\partial_x  )f\left(r\right)&=\mqty(\partial_xf\left(r\right)\\\partial_yf\left(r\right) \\\partial_xf\left(r\right)  )\\
&=\mqty(\frac{\partial fr}{\partial x} \ \frac{\partial r}{\partial r}\\\frac{\partial fr}{\partial y} \ \frac{\partial r}{\partial r}\\\frac{\partial fr}{\partial z} \ \frac{\partial r}{\partial r})\\
&=\mqty(\frac{\partial fr}{\partial r} \ \frac{\partial r}{\partial x}\\\frac{\partial fr}{\partial r} \ \frac{\partial r}{\partial y}\\\frac{\partial fr}{\partial r} \ \frac{\partial r}{\partial z})\\
&=\mqty(\frac{\partial fr}{\partial r} \ \frac{x}{r}\\\frac{\partial fr}{\partial r} \ \frac{y}{r}\\\frac{\partial fr}{\partial r} \ \frac{z}{r})\\
&=\mqty(\frac{\partial fr}{\partial r} \\\frac{\partial fr}{\partial r} \\\frac{\partial fr}{\partial r})\cdot \hat{\vec{r}}\\
&=\frac{\text{d}f}{\text{d}r}\hat{\vec{r}}
\end{align*}
\subsection*{c)}$\text{rot }\vec{r}=\vec{\nabla}\times\vec{r}$
\begin{align*}
\mqty(\partial_x\\\partial_y\\\partial_z)\times\mqty(x\\y\\z)&=\mqty(\frac{\partial z}{\partial y}-\frac{\partial y}{\partial z}\\\frac{\partial x}{\partial z}-\frac{\partial z}{\partial x}\\\frac{\partial y}{\partial x}-\frac{\partial x}{\partial y})\\
&=\mqty(0\\0\\0)
\end{align*}
\subsection*{d)}$\text{div }\vec{r}=\vec{\nabla}\cdot\vec{r}$
\begin{align*}
\mqty(\partial_x\\\partial_y\\\partial_z)\cdot\mqty(x\\y\\z)&=\frac{\partial x}{\partial x}+\frac{\partial y}{\partial y}+\frac{\partial z}{\partial z}\\
&=1+1+1\\
&=3
\end{align*}

\newpage
\section*{Aufgabe 4.2}


\newpage
\section*{Aufgabe 4.3}Ein Massenpunkt wird unter dem Winkel $\theta$ "uber der Horizontalen von einer Klippe geworfen.Zum Zeitpunkt $t= 0$ sei der Massenpunkt am Ort $\vec{r}\left(0\right) = \left(0,0,0\right)^\text{T}$ und der Betrag der Geschwindigkeit sei $\left|\vec{v}\left(0\right)\right|=v_0$. Wir nehmen ein viskoses Medium mit Stokesscher Reibung an, indem die Beschleunigung durch$\vec{a}=-k\vec{v}-g\vec{e_z}$ gegeben ist. Durch Integration der Gleichung
\begin{align*}
\vec{a}\left(t\right)=\frac{\text{d}}{\text{d}t}\vec{v}\left(t\right)=\frac{\text{d}^2}{\text{d}t^2}\vec{r}\left(t\right)
\end{align*}
erhalten Sie zun"achst $\vec{v}\left(t\right)$ und dann $\vec{r}\left(t\right)$. Die jeweiligen Integrationskonstanten sind durch die Vorgabe  $\vec{v}\left(t\right)= v_0\left(\cos \theta,0,\sin\theta\right)^\text{T}$ und $\vec{r}$ festgelegt.
\subsection*{a)} Bestimmen Sie die Geschwindigkeit des Massenpunktes als Funktion der Zeit.
\begingroup
\allowdisplaybreaks
\begin{align*}
\vec{a}&=-k\vec{v}-g\vec{e_z}\\
\vec{a}&=\dot{\vec{v}}\\
\vec{a}&=\frac{\text{d}\vec{v}}{\text{d}t}\\\\
\text{Substitution} \ \vec{v}=\vec{\phi}\left(t\right)e^{-kt}&=\vec{\phi}e^{-kt}\\
\dot{\vec{v}}&=\dot{\vec{\phi}}e^{-kt}-k\vec{\phi}e^{-kt}\\\\
\text{Nach }\dot{\vec{\phi}}\text{ umformen:}\\
\dot{\vec{v}}&=\dot{\vec{\phi}}e^{-kt}-k\vec{\phi}e^{-kt}&&|+k\vec{\phi}e^{-kt}\\
\dot{\vec{\phi}}e^{-kt}&=\dot{\vec{v}}+k\vec{\phi}e^{-kt}&&|\dot{\vec{v}}=\vec{a}=-k\vec{v}-g\vec{e_z}\\
\dot{\vec{\phi}}e^{-kt}&=-k\vec{v}-g\vec{e_z}+k\vec{\phi}e^{-kt}&&|\cdot e^{kt}\\
\dot{\vec{\phi}}&=-g\vec{e_z}e^{kt}\\\\
\text{Durch Integration von }\dot{\vec{\phi}}\text{ l"asst sich }\vec{\phi}\text{ berechnen:}\\
\vec{\phi}&=\int\dot{\vec{\phi}} \ \text{d}t\\
&=\int\text{d}t-g\vec{e_z}e^{kt}\\
&=-\frac{g}{k}\vec{e_z}e^{kt}+\vec{\phi_c}\\\\
\text{Resubstitution} \ \vec{v}\\
\vec{v}&=\vec{\phi}e^{-kt}\\
&=\left(-\frac{g}{k}\vec{e_z}e^{kt}+\vec{\phi_c}\right)e^{-kt}\\
&=-\frac{g}{k}+\vec{\phi_c}e^{-kt}\\\\\\\\
\text{Nun muss nur noch} \ \vec{\phi_c} \ \text{bestimmt werden.}\\\\
\text{Bestimmung von} \ \vec{\phi_c} \ \text{durch Startbedingung} \ \left|\vec{v}\left(0\right)\right|&=v_0\\
\vec{v}\left(0\right)&=\mqty(\phi_c^xe^{-0k}\\\phi_c^ye^{-0k}\\\phi_c^ze^{-0k}-\frac{g}{k})\\
&=\mqty(\phi_c^x\\\phi_c^y\\\phi_c^z-\frac{g}{k})=\mqty(v_0\cos\theta\\0\\v_0\sin\theta)\\
&\Rightarrow\mqty(\phi_c^x\\\phi_c^y\\\phi_c^z)=\mqty(v_0\cos\theta\\0\\v_0\sin\theta+\frac{g}{k})\\\\
\text{Somit ergibt sich f"ur} \ \vec{v}\left(t\right)\text{:}\\
\vec{v}\left(t\right)&=\mqty(v_0\cdot e^{-kt}\cos\theta\\0\\\left(v_0\sin\theta+\frac{g}{k}\right)e^{-kt}-\frac{g}{k})
\end{align*}
\endgroup
\newpage
\subsection*{b)}Bestimmen Sie die Bahnkurve $\vec{r}\left(t\right)$ des Massenpunktes.\\\\
Die Bahnkurve l"asst sich durch einfaches Integrieren der Geschwindigkeit berechnen.\\
Aus "ubersichtsgr"unden werden wir die $x$, $y$ und $z$ Komponenten einzeln integrieren.\\\\
F"ur $x$:
\begin{align*}
r^x\left(t\right)&=\int\text{d}t \ v^x\\
&=\int\text{d}t \ v_0\cdot e^{-kt}\cdot \cos\theta\\
&=-\frac{v_0}{k} e^{-kt}\cdot \cos\theta+r_c^x
\end{align*}
F"ur $y$:
\begin{align*}
r^y\left(t\right)&=\int\text{d}t \ v^y\\
&=\int\text{d}t \ 0\\
&=r_c^y
\end{align*}
F"ur $z$:
\begin{align*}
r^z\left(t\right)&=\int\text{d}t \ v^z\\
&=\int\text{d}t \ \left(v_0\sin\theta +\frac{g}{k}\right)e^{-kt}-\frac{g}{k}\\
&=-\frac{v_0\sin\theta +\frac{g}{k}}{k}e^{-kt}-\frac{g}{k}t+r_c^z
\end{align*}
Als Zwischenergebnis f"ur $\vec{r}\left(t\right)$ ergibt sich:
\begin{align*}
\vec{r}\left(t\right)=\mqty(-\frac{v_0}{k} e^{-kt}\cdot \cos\theta+r_c^x \\ r_c^y \\ -\frac{v_0\sin\theta +\frac{g}{k}}{k}e^{-kt}-\frac{g}{k}t+r_c^z )
\end{align*}
Nun muss muss $\vec{r_c}$ nur noch an die Anfangsbedingungen angepasst werden:
\begin{align*}
\vec{r}\left(0\right)&=\mqty(-\frac{v_0}{k} e^{-0k}\cdot \cos\theta+r_c^x \\ r_c^y \\ -\frac{v_0\sin\theta +\frac{g}{k}}{k}e^{-0k}-\frac{g}{k}\cdot 0+r_c^z )=\mqty(0\\0\\0)\\
&\Rightarrow\mqty(r_c^x\\r_c^y\\r_c^z)=\mqty(\frac{v_0}{k}\cdot \cos\theta\\0\\\frac{v_0\sin\theta +\frac{g}{k}}{k})
\end{align*}
Somit ergibt sich f"ur $\vec{r}\left(t\right)$:
\begin{align*}
\vec{r}\left(t\right)&=\mqty(-\frac{v_0}{k} e^{-kt}\cdot \cos\theta+\frac{v_0}{k}\cdot \cos\theta \\ 0 \\ -\frac{v_0\sin\theta +\frac{g}{k}}{k}e^{-kt}-\frac{g}{k}t+\frac{v_0\sin\theta +\frac{g}{k}}{k} )\\\\
\text{Vereinfacht ergibt dies:}\\
\vec{r}\left(t\right)&=\mqty(\left(1-e^{-kt}\right)\frac{v_0}{k}\cdot \cos\theta\\0\\\left(1-e^{-kt}\right)\frac{v_0\sin\theta +\frac{g}{k}}{k}-\frac{g}{k}t)
\end{align*}
\newpage
\subsection*{c)} Zeigen Sie, dass der Massenpunkt im Limes $t$ $\rightarrow$ $\infty$ eine endliche Grenzgeschwindigkeit erreicht und berechnen Sie deren Betrag. Wie weit kommt der Massenpunkt maximal in $x^1$-Richtung? ($x^1$-Richtung entspricht hier $x$-Richtung)\\
Als erstes berechnen wir den Vektor im Limes $t$ $\rightarrow$ $\infty$:
\begin{align*}
\lim_{t\rightarrow\infty}\vec{v}\left(t\right)&=\lim_{t\rightarrow\infty}\mqty(v_0\cdot e^{-kt}\cos\theta\\0\\\left(v_0\sin\theta+\frac{g}{k}\right)e^{-kt}-\frac{g}{k})\\\\
\text{Da} \ \lim_{t\rightarrow\infty}e^{-kt}=0 \ \text{:}&\\
\lim_{t\rightarrow\infty}\vec{v}\left(t\right)&=\mqty(0\\0\\-\frac{g}{k})\\\\
\text{Nun berechnen wir den Betrag der Geschwindgkeit:}&\\
\left|\lim_{t\rightarrow\infty}\vec{v}\left(t\right)\right|&=\left|\mqty(0\\0\\-\frac{g}{k})\right|\\
\left|\lim_{t\rightarrow\infty}\vec{v}\left(t\right)\right|&=\sqrt{\left(\frac{g}{k}\right)^2}\\
\left|\lim_{t\rightarrow\infty}\vec{v}\left(t\right)\right|&=\frac{g}{k}
\end{align*}
Betrachten wir die zur"uckgelegte Strecke in $x$-Richtung im Limes $t$ $\rightarrow$ $\infty$:
\begin{align*}
\lim_{t\rightarrow\infty}r^x\left(t\right)&=\lim_{t\rightarrow\infty}\left(1-e^{-kt}\right)\frac{v_0}{k}\cdot \cos\theta\\
\lim_{t\rightarrow\infty}r^x\left(t\right)&=\frac{v_0}{k}\cdot \cos\theta
\end{align*}
Dies ist die in $x$-Richtung maximal zur"uckgelegte Entfernung.
\newpage
\subsection*{d)} Zeigen Sie: Der h"ochste Bahnpunkt wird nach der Zeit
\begin{align*}
T=\frac{1}{k}\ln\left(1+\frac{v_0k\sin\theta}{g}\right)
\end{align*}
erreicht.\\\\
Der h"ochste Bahnpunkt h"angt nur von der $z$-Richtung ab. Da wir die Bahnkurve $\vec{r}\left(t\right)$ betrachten, m"ussen wir nur das Maximum der $z$ Richtung berechnen. An dieser Stelle ist $v^z\left(t\right)=0$:
\begin{align*}
v^z\left(t\right)&=0\\
\left(v_0\sin\theta +\frac{g}{k}\right)e^{-kt}-\frac{g}{k}&=0&&|+\frac{g}{k}\\
\left(v_0\sin\theta +\frac{g}{k}\right)e^{-kt}&=\frac{g}{k}&&|\cdot \frac{1}{v_0\sin\theta +\frac{g}{k}}\\
e^{-kt}&=\frac{g}{k\left(v_0\sin\theta +\frac{g}{k}\right)}&&|\cdot e^{kt}\cdot\frac{k\left(v_0\sin\theta +\frac{g}{k}\right)}{g} \\
e^{kt}&=\frac{k\left(v_0\sin\theta +\frac{g}{k}\right)}{g}&&|\text{Vereinfachen}\\
e^{kt}&=\frac{kv_0\sin\theta}{g}+1&&|\text{nach} \ t \ \text{aufl"osen} \ \Rightarrow \ln\\
\ln\left(e^{kt}\right)&=\ln\left(\frac{kv_0\sin\theta}{g}+1\right)&&|\cdot\frac{1}{k}\\
t&=\frac{1}{k}\ln\left(1+\frac{v_0k\sin\theta}{g}\right)
\end{align*}
Dies ist der zu zeigende Term.
\newpage
\subsection*{e)} Zeigen Sie: Die H"ohe am Scheitelpunkt der Bahn ist
\begin{align*}
H=\frac{v_0\sin\theta}{k}-\frac{g}{k^2}\ln\left(1+\frac{v_0k\sin\theta}{g}\right)\text{.}
\end{align*}\\
Setzen wir nun den in \textbf{d)} berechneten Zeitpunkt f"ur $r^z\left(t\right)$ ein, so bekommen wir die maximale H"ohe der Bahnkurve, diese ist dann auch die H"ohe des Scheitelpunktes:
\begin{align*}
r^z\left(t\right)&=\left(1-e^{-kt}\right)\frac{v_0\sin\theta +\frac{g}{k}}{k}-\frac{g}{k}t
\end{align*}
Da der Term durch einsetzen von $T$ noch viel gr"o"ser und un"ubersichtlicher wird, wollen wir zuerst nur  $e^{-kT}$ und $\frac{g}{k}T$ betrachten:\\\\
F"ur $e^{-kT}$:
\begin{align*}
e^{-kT}&=e^{-k\cdot\frac{1}{k}\ln\left(1+\frac{v_0k\sin\theta}{g}\right)}\\
&=e^{-\ln\left(1+\frac{v_0k\sin\theta}{g}\right)}\\
&=\frac{1}{e^{\ln\left(1+\frac{v_0k\sin\theta}{g}\right)}}\\
&=\frac{1}{1+\frac{v_0k\sin\theta}{g}}\\
&=\frac{1+\frac{v_0k\sin\theta}{g}}{1+\frac{v_0k\sin\theta}{g}}+\frac{-\frac{v_0k\sin\theta}{g}}{1+\frac{v_0k\sin\theta}{g}}&&|\text{Hinteren Summanden mit} \ \frac{\frac{g}{k}}{\frac{g}{k}} \ \text{erweitern.}\\
&=1-\frac{\frac{g}{k}\cdot\frac{k}{g}v_0\sin\theta}{\frac{g}{k}+\frac{g}{k}\cdot\frac{k}{g}v_0\sin\theta}&&|\text{k"urzen!}\\
&=1-\frac{v_0\sin\theta}{\frac{g}{k}+v_0\sin\theta}
\end{align*}
F"ur $\frac{g}{k}T$:
\begin{align*}
\frac{g}{k}T&=\frac{g}{k}\cdot\frac{1}{k}\ln\left(1+\frac{v_0k\sin\theta}{g}\right)\\
&=\frac{g}{k^2}\cdot\ln\left(1+\frac{v_0k\sin\theta}{g}\right)
\end{align*}
Setzen wir die zwei berechneten Gleichungen nun in $r^z$ ein so ergibt dies:
\begin{align*}
r^z\left(T\right)&=\left(1-\left(1-\frac{v_0\sin\theta}{\frac{g}{k}+v_0\sin\theta}\right)\right)\frac{v_0\sin\theta +\frac{g}{k}}{k}-\frac{g}{k^2}\cdot\ln\left(1+\frac{v_0k\sin\theta}{g}\right)\\
r^z\left(T\right)&=\left(\frac{v_0\sin\theta}{\frac{g}{k}+v_0\sin\theta}\right)\frac{v_0\sin\theta +\frac{g}{k}}{k}-\frac{g}{k^2}\cdot\ln\left(1+\frac{v_0k\sin\theta}{g}\right)\\
r^z\left(T\right)&=\frac{v_0\sin\theta}{k}-\frac{g}{k^2}\ln\left(1+\frac{v_0k\sin\theta}{g}\right)
\end{align*}
Dies ist der zu zeigende Term.
\newpage
\subsection*{f)}Zeigen Sie: F"ur kleine Reibung ($k$ $\rightarrow$ 0) sind die Geschwindigkeit $\vec{v}\left(t\right)$ und der Ortsvektor $\vec{r}\left(t\right)$ ann"ahernd gegeben durch
\begin{align*}
\vec{v}\left(t\right)=\vec{v_0}-gt\vec{e_z}\text{,}&&\vec{r}\left(t\right)=t\vec{v_0}-\frac{1}{2}gt^2\vec{e_z}\text{.}
\end{align*}
F"ur $\vec{v}\left(t\right)$ haben wir in \textbf{a)} berechnet:
\begin{align*}
\vec{v}\left(t\right)&=\mqty(v_0\cdot e^{-kt}\cos\theta\\0\\\left(v_0\sin\theta+\frac{g}{k}\right)e^{-kt}-\frac{g}{k})\\\\
\vec{v}\left(0\right)&=\mqty(v_0\cos\theta\\0\\v_0\sin\theta)
\end{align*}
Im Limes der Reibung ($k$ $\rightarrow$ 0) bedeutet dies:
\begin{align*}
\lim_{k\rightarrow 0}\vec{v}\left(t\right)&=\lim_{k\rightarrow 0}\mqty(v_0\cdot e^{-kt}\cos\theta\\0\\\left(v_0\sin\theta+\frac{g}{k}\right)e^{-kt}-\frac{g}{k})\\
\text{Da f"ur} \ k \ \rightarrow \ 0\text{,} &\  e^{-kt} \ \text{in die Taylorreihe} \ 1-kt+\frac{1}{2}k^2t^2 \ \text{entwickelt werden kann, gilt:} \\
\lim_{k\rightarrow 0}\vec{v}\left(t\right)&=\lim_{k\rightarrow 0}\mqty(v_0\cdot \left(1-kt+\frac{1}{2}k^2t^2\right)\cdot\cos\theta\\0\\\left(v_0\sin\theta+\frac{g}{k}\right)\cdot\left(1-kt+\frac{1}{2}k^2t^2\right)-\frac{g}{k})\\
&=\lim_{k\rightarrow 0}\mqty(v_0\cdot \left(1-kt+\frac{1}{2}k^2t^2\right)\cdot\cos\theta\\0\\v_0\sin\theta\cdot\left(1-kt+\frac{1}{2}k^2t^2\right) )+\lim_{k\rightarrow 0}\mqty(0\\0\\\frac{g}{k}\cdot\left(1-kt+\frac{1}{2}k^2t^2\right)-\frac{g}{k})\\
&=\mqty(v_0\cdot\cos\theta\\0\\v_0\cdot\sin\theta)+\lim_{k\rightarrow 0}\mqty(0\\0\\\frac{g}{k}-kt\cdot\frac{g}{k}+\frac{1}{2}k^2t^2\cdot\frac{g}{k}-\frac{g}{k})\\
&=\mqty(v_0\cdot\cos\theta\\0\\v_0\cdot\sin\theta)+\lim_{k\rightarrow 0}\mqty(0\\0\\-tg+\frac{1}{2}kt^2\cdot g)\\
&=\mqty(v_0\cdot\cos\theta\\0\\v_0\cdot\sin\theta)+\mqty(0\\0\\-tg)\\
&=\vec{v_0}-gt\vec{e_z}
\end{align*}\\
\newpage
F"ur $\vec{r}\left(t\right)$ haben wir in \textbf{b)} berechnet:
\begin{align*}
\vec{r}\left(t\right)&=\mqty(\left(1-e^{-kt}\right)\frac{v_0}{k}\cdot \cos\theta\\0\\\left(1-e^{-kt}\right)\frac{v_0\sin\theta +\frac{g}{k}}{k}-\frac{g}{k}t)
\end{align*}
Im Limes der Reibung ($k$ $\rightarrow$ 0) bedeutet dies:
\begin{align*}
\lim_{k\rightarrow 0}\vec{r}\left(t\right)&=\lim_{k\rightarrow 0}\mqty(\left(1-e^{-kt}\right)\frac{v_0}{k}\cdot \cos\theta\\0\\\left(1-e^{-kt}\right)\frac{v_0\sin\theta +\frac{g}{k}}{k}-\frac{g}{k}t)\\
\text{Da f"ur} \ k \ \rightarrow \ 0\text{,} &\  e^{-kt} \ \text{in die Taylorreihe} \ 1-kt+\frac{1}{2}k^2t^2 \ \text{entwickelt werden kann, gilt:} \\
\lim_{k\rightarrow 0}\vec{r}\left(t\right)&=\lim_{k\rightarrow 0}\mqty(\left(1-\left(1-kt+\frac{1}{2}k^2t^2\right)\right)\frac{v_0}{k}\cdot \cos\theta\\0\\\left(1-\left(1-kt+\frac{1}{2}k^2t^2\right)\right)\frac{v_0\sin\theta +\frac{g}{k}}{k}-\frac{g}{k}t)\\
&=\lim_{k\rightarrow 0}\mqty(\left(kt-\frac{1}{2}k^2t^2\right)\frac{v_0}{k}\cdot\cos\theta\\0\\\left(kt-\frac{1}{2}k^2t^2\right)\frac{v_0\sin\theta+\frac{g}{k}}{k}-\frac{g}{k}t)\\
&=\lim_{k\rightarrow 0}\mqty(tv_0\cos\theta-\frac{1}{2}kt^2 v_0\cos\theta\\0\\ tv_0 \sin\theta +\frac{g}{k}t-\frac{1}{2}kt^2\left(v_0\sin\theta+\frac{g}{k}\right)-\frac{g}{k}t)\\
&=\mqty(tv_0\cos\theta\\0\\tv_0 \sin\theta)+\lim_{k\rightarrow 0}\mqty(\frac{1}{2}kt^2 v_0\cos\theta\\0\\-\frac{1}{2}kt^2\left(v_0\sin\theta+\frac{g}{k}\right))\\
&=\mqty(tv_0\cos\theta\\0\\tv_0 \sin\theta)+\mqty(0\\0\\-\frac{1}{2}gt^2)+\lim_{k\rightarrow 0}\mqty(\frac{1}{2}kt^2 v_0\cos\theta\\0\\-\frac{1}{2}kt^2v_0\sin\theta)\\
&=\mqty(tv_0\cos\theta\\0\\tv_0 \sin\theta)+\mqty(0\\0\\-\frac{1}{2}gt^2)\\
&=t\vec{v_0}-\frac{1}{2}gt^2\vec{e_z}
\end{align*}
\newpage
\section*{Aufgabe 4.4}
Wir bestimmen eine Stammfunktion von 
	\[
		f(x) = \frac{1}{a+bx+cx^2}, \ \ \ \ \ b^{2} > 4ac.
	\]
Für die Nullstellen von $a+bx+cx^{2}$ gilt:
	\[
		x_{1} = \frac{-b + \sqrt{b^{2}-4ac}}{2c}, \ \ \ \ \ x_{2} = \frac{-b - \sqrt{b^{2}-4ac}}{2c}.
	\]
Wir wählen den Ansatz wie gegeben:
	\[
		\frac{1}{a+bx+cx^2} = \frac{\alpha}{x-x_{1}} + \frac{\beta}{x-x_{2}}
	\]
und erhalten:
	\[
		0 \cdot x + 1 = (\alpha + \beta) \cdot x + (- \alpha x_{2} - \beta x_{1})
	\]
Koeffizientenvergleich liefert:
	\begin{align*}
		0 &= \alpha + \beta, \\
		1 &= -\alpha x_{2} - \beta x_{1}.
	\end{align*}
Woraus wir erhalten $\alpha = - \beta$ und somit:
	\[
		\alpha (x_{1} - x_{2}) = 1, \ \ \ \implies \alpha = \frac{1}{x_{1} - x_{2}} = \frac{c}{\sqrt{b^{2}-4ac}}.
	\]
Somit erhalten wir eine Stammfunktion von $f$ durch:
	\begin{align*}
		\int f \dd x &= \int \frac{1}{a+bx+cx^2} \dd x \\
		&= \int \frac{\alpha}{x-x_{1}} + \frac{\beta}{x-x_{2}} \dd x \\
		&= \alpha \ln(x-x_{1}) + \beta \ln(x-x_{2}) \\
		&= \alpha \qty(\ln(x-x_{1}) - \ln(x-x_{2})) \\
		&= \frac{c}{\sqrt{b^{2}-4ac}} \qty[ \ln(x - \frac{-b + \sqrt{b^{2}-4ac}}{2c}) - \ln(x + \frac{+b + \sqrt{b^{2}-4ac}}{2c}) ]
	\end{align*}


\newpage
\section*{Aufgabe 4.5}









\end{document}
