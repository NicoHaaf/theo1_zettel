\documentclass{theozettel}

%%%%%%%%%%%%%%%%%%%%%%%%%%%%%%%%%%%%%%%%%%%%%%%%%%%%%%%%%%%%%%%%%%%%%%%%%%%%%%%%%%%%%%%%%%%%%%%%%%%%%%%%%%%%%%
% page geometry
%%%%%%%%%%%%%%%%%%%%%%%%%%%%%%%%%%%%%%%%%%%%%%%%%%%%%%%%%%%%%%%%%%%%%%%%%%%%%%%%%%%%%%%%%%%%%%%%%%%%%%%%%%%%%%
\geometry{
	left=20mm,
	right=20mm,
	top=25mm,
	bottom=20mm
}
%%%%%%%%%%%%%%%%%%%%%%%%%%%%%%%%%%%%%%%%%%%%%%%%%%%%%%%%%%%%%%%%%%%%%%%%%%%%%%%%%%%%%%%%%%%%%%%%%%%%%%%%%%%%%%

\pgfplotsset{compat=1.16}

\usepackage{dsfont}

\theoI{2}

\renewcommand{\epsilon}{\varepsilon}

\begin{document}
\punkteIV{2.1}{2.2}{2.3}{2.4}

\section*{Aufgabe 2.1}
\begin{enumerate}[(a)]
	\item 	Wir betrachten $\vec{x} = (x_{1}, 0, 0)$ und $\vec{y}=(y_{1},y_{2},0)$ mit $x_{1}, y_{1}, y_{2} > 0$. Dann gilt für $\vec{z} = \vec{x} \times \vec{y}$:
			\[
				z_{i} = \sum_{j=1}^{3}\sum_{k=1}^{3} \epsilon^{ijk}x_{j}y_{j} \stackrel{x_{2}=x_{3}=0}{=} \underbrace{\epsilon^{i11}}_{= 0}x_{1}y_{1} + \epsilon^{i12}x_{1}y_{2} + \epsilon^{i13}x_{1}\underbrace{y_{3}}_{= 0} = \epsilon^{i12}x_{1}y_{2}.
			\]
			Mit der Definition des Levi-Civita-Symbols erhalten wir also:
			\[
				z_{1} = z_{2} = 0, \ \ \ \ \  z_{3} = x_{1}y_{2}.
			\]
			$\vec{x}$ und $\vec{y}$ liegen in der Ebene gegeben durch alle $\vec{a}$ mit $a_{3} = 0$ und der Winkel von $\theta$ zwischen $\vec{x}$ und $\vec{y}$ ist genau der Winkel zwischen $\vec{y}$ und der ersten Koordinatenachse. Dies wird an folgender Abbildung klar:
			\begin{figure}[h]
			\centering
			\begin{tikzpicture}
				\draw[->] (-0.5,0) -- (6,0);
				\draw[->] (0,-0.5) -- (0,4);
				
				\draw[->, thick, blue] (0,0) -- (5.3,0);
				\node at (4.5,0) [label=below:\textcolor{blue}{$\vec{x}$}] {};
				
				\draw[->, thick, red] (0,0) -- (4,3);
				\node at (2,1.5) [label=left:\textcolor{red}{$\vec{y}$}] {};
				
				\draw[thick, red] (0,0) -- (4,0);
				\node at (2,0) [label=below:\textcolor{red}{$y_{1}$}] {};
				
				\draw[thick, red] (4,0) -- (4,3);
				\node at (4,1.5) [label=right:\textcolor{red}{$y_{2}$}] {};
			\end{tikzpicture}
			\end{figure}
			
			Dann sehen wir geometrisch ein:
			\[
				\sin(\theta) = \frac{y_{2}}{\qty|\vec{y}|}
			\]
			Woraus wir direkt folgern:
			\[
				\qty|\vec{z}| = x_{1}y_{2} =x_{1} \cdot y_{2} \cdot \frac{\qty|\vec{y}|}{\qty|\vec{y}|} = \qty|x| \cdot \qty|y| \cdot \sin(\theta).
			\]
			Ferner folgt $\vec{z} \perp \vec{x}$ und $\vec{z} \perp \vec{y}$ durch:
			\begin{align*}
				\vec{z} \cdot \vec{x} &= x_{1}z_{1} + x_{2}z_{2} + x_{3}z_{3} = x_{1} \cdot 0 + 0 \cdot 0 + 0 \cdot z_{3} = 0, \\
				\vec{z} \cdot \vec{y} &= y_{1}z_{1} + y_{2}z_{2} + y_{3}z_{3} = y_{1} \cdot 0 + y_{2} \cdot 0 + 0 \cdot z_{3} = 0.
			\end{align*}
			
	\item 	Es ist $\epsilon^{ijk} = 0$ genau dann wenn zwei Indizes gleich sind. In diesem Fall ist die Antisymmetrie klar. Nun seien $i,j,k$ also verschieden. Dann wird die Antisymmetrie klar, durch aufzählen aller möglichen Fälle:
			\begin{align*}
				\epsilon^{123} = 1 &&& \longleftrightarrow &&& \epsilon^{213} = -1 \\
				&&& \longleftrightarrow &&& \epsilon^{321} = -1 \\
				&&& \longleftrightarrow &&& \epsilon^{132} = -1 \\
				\epsilon^{312} = 1 &&& \longleftrightarrow &&& \epsilon^{132} = -1 \\
				&&& \longleftrightarrow &&& \epsilon^{213} = -1 \\
				&&& \longleftrightarrow &&& \epsilon^{321} = -1 \\
				\epsilon^{231} = 1 &&& \longleftrightarrow &&& \epsilon^{321} = -1 \\
				&&& \longleftrightarrow &&& \epsilon^{132} = -1 \\
				&&& \longleftrightarrow &&& \epsilon^{213} = -1 \\
			\end{align*}
			Wir folgern:
			\begin{align*}
				\vec{x} \cdot (\vec{x} \times \vec{y}) &= \mqty(x_{1} \\ x_{2} \\ x_{3}) \cdot \mqty( \sum_{j=1}^{3}\sum_{k=1}^{3} \epsilon^{1jk}x_{j}y_{k}2 \\ \sum_{j=1}^{3}\sum_{k=1}^{3} \epsilon^{2jk}x_{j}y_{k} \\ \sum_{j=1}^{3}\sum_{k=1}^{3} \epsilon^{3jk}x_{j}y_{k}) = \mqty(x_{1} \\ x_{2} \\ x_{3}) \cdot \mqty( \epsilon^{123}x_{2}y_{3} + \epsilon^{132}x_{3}y_{2} \\ \epsilon^{213}x_{1}y_{3} + \epsilon^{231}x_{3}y_{1} \\ \epsilon^{312}x_{1}y_{2} + \epsilon^{321}x_{2}y_{1} ) \\		
				&= x_{1}\epsilon^{123}x_{2}y_{3} +x_{2}\epsilon^{213}x_{1}y_{3} + x_{1}\epsilon^{132}x_{3}y_{2} + x_{3}\epsilon^{312}x_{1}y_{2} + x_{2}\epsilon^{231}x_{3}y_{1} + x_{3}\epsilon^{321}x_{2}y_{1} \\
				&= 	\epsilon^{123}x_{1}x_{2}y_{3} - \epsilon^{123}x_{1}x_{2}y_{3} + \epsilon^{132}x_{1}x_{3}y_{2} -\epsilon^{132}x_{1}x_{3}y_{2} + \epsilon^{231}x_{2}x_{3}y_{1} - \epsilon^{231}x_{2}x_{3}y_{1} \\ &= 0.
			\end{align*}
			Vollkommen analog rechnen wir $\vec{y} \cdot (\vec{x} \times \vec{y})$. 
\end{enumerate}


\newpage
\section*{Aufgabe 2.2}


\newpage
\section*{Aufgabe 2.3}

a)

Der Geschwindigkeitsvektor ergibt sich aus der Ableitung des Positionsvektors:

$$
\vec{x}(t) = \begin{pmatrix}
R cos(\omega t) \\
R sin(\omega t) \\
v_0 t
\end{pmatrix}
$$

$$
\vec{v	}(t) = \dot{vec{x}}(t) = \begin{pmatrix}
- R \omega sin(\omega t) \\
R \omega cos(\omega t) \\
v_0
\end{pmatrix}
$$

Die Bogenlänge berechnet sich wie folgt:

$$
s(t) =  \int_{t_0}^{t} d t' | \vec{v}(t') | = \int_{t_0}^{t} d t' \sqrt{\vec{v}(t')^{2}} = \int_{0}^{t} d t' \sqrt{R^{2}\omega^{2} ((-sin(\omega t'))^{2} + cos(\omega t')^{2} + v_0^{2}} = \sqrt{R^{2} \omega^{2} + v_0^{2}} t
$$

Man beachte hierbei, dass die Integrationskonstante tatsächlich $0$ ist, da $s(t_0 = 0) = 0$ angenommen wurde.

Die Formel kann umgestellt werden zu:

$$
t(s) = \frac{s}{\sqrt{R^{2} \omega^{2} + v_0^{2}}}
$$

Der Tangentenvektor entspricht dem auf $1$ normierten Geschwindigkeitsvektor:

$$
\vec{T}(s) = \frac{\vec{v}(t(s))}{|\vec{v}(t(s))|} = \frac{1}{\sqrt{R^{2} \omega^{2} + v_0^{2}}} \begin{pmatrix}
- R \omega sin(\frac{\omega s}{\sqrt{R^{2} \omega^{2} + v_0^{2}}}) \\
R \omega cos(\frac{\omega s}{\sqrt{R^{2} \omega^{2} + v_0^{2}}}) \\
v_0
\end{pmatrix}
$$

Für den Krümmungsradius gilt:

$$
\rho = \frac{1}{|\frac{d\vec{T}}{ds}|} = \frac{1}{|\begin{pmatrix}
\frac{- R \omega^{2}}{\sqrt{R^{2} \omega^{2} + v_0^{2}}}cos(\frac{\omega s}{\sqrt{R^{2} \omega^{2} + v_0^{2}}}) \\
\frac{- R \omega^{2}}{\sqrt{R^{2} \omega^{2} + v_0^{2}}}sin(\frac{\omega s}{\sqrt{R^{2} \omega^{2} + v_0^{2}}}) \\
0
\end{pmatrix} |} = \frac{1}{\frac{- R \omega^{2}}{\sqrt{R^{2} \omega^{2} + v_0^{2}}}}
$$

Der Normalenvektor ergibt sich wie folgt:

$$
\vec{N}(s) = \rho \frac{d\vec{T}}{ds} =\begin{pmatrix}
-cos(\frac{\omega s}{\sqrt{R^{2} \omega^{2} + v_0^{2}}}) \\
-sin(\frac{\omega s}{\sqrt{R^{2} \omega^{2} + v_0^{2}}}) \\
0
\end{pmatrix} 
$$

Der Binormalenvektor berechnet sich wie folgt:

\begin{align}
\vec{B}(s) &= \vec{T}(s) \times \vec{N}(s) \\&= \frac{1}{\sqrt{R^{2} \omega^{2} + v_0^{2}}} \begin{pmatrix}
R \omega cos(\omega t(s)) 0 - v_0 (-sin(\omega t(s))) \\
v_0 -cos(\omega t(s)) - 0 (-R \omega sin(\omega t(s)) \\
-R \omega sin(\omega t(s) (-sin(\omega t(s)))  - R \omega \cos(\omega t(s)) (-cos(\omega t(s)))
\end{pmatrix} \\ &= \frac{1}{\sqrt{R^{2} \omega^{2} + v_0^{2}}} \begin{pmatrix}
v_0 (-sin(\omega \frac{s}{\sqrt{R^{2} \omega^{2} + v_0^{2}}})) \\
v_0 (-cos(\omega \frac{s}{\sqrt{R^{2} \omega^{2} + v_0^{2}}})) \\
R \omega
\end{pmatrix}
\end{align}
b)

Aufgrund der alternativen Definition des Vektorprodukts muss gelten, dass $\vec{B}(s) \perp \vec{T}(s)$ und $\vec{B}(s) \perp \vec{N}(s)$. Dass $\vec{B}(s) \perp \vec{N}(s)$ gilt, kann dadurch gezeigt werden, dass das Skalarprodukt $0$ ist:

\begin{align}
\vec{T}(s) \cdot \vec{N}(s) &= \frac{1}{\sqrt{R^{2} \omega^{2} + v_0^{2}}}
((- R \omega sin(\frac{\omega s}{\sqrt{R^{2} \omega^{2} + v_0^{2}}})) (-cos(\frac{\omega s}{\sqrt{R^{2} \omega^{2} + v_0^{2}}})) \\&+ (R \omega cos(\frac{\omega s}{\sqrt{R^{2} \omega^{2} + v_0^2}})) (-sin(\frac{\omega s}{\sqrt{R^{2} \omega^{2}
+ v_0^{2}}})) + 0 v_0 = 0
\end{align}




\newpage
\section*{Aufgabe 2.4}











\end{document}