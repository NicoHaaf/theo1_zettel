\documentclass{theozettel}

%%%%%%%%%%%%%%%%%%%%%%%%%%%%%%%%%%%%%%%%%%%%%%%%%%%%%%%%%%%%%%%%%%%%%%%%%%%%%%%%%%%%%%%%%%%%%%%%%%%%%%%%%%%%%%
% page geometry
%%%%%%%%%%%%%%%%%%%%%%%%%%%%%%%%%%%%%%%%%%%%%%%%%%%%%%%%%%%%%%%%%%%%%%%%%%%%%%%%%%%%%%%%%%%%%%%%%%%%%%%%%%%%%%
\geometry{
	left=20mm,
	right=20mm,
	top=25mm,
	bottom=20mm
}
%%%%%%%%%%%%%%%%%%%%%%%%%%%%%%%%%%%%%%%%%%%%%%%%%%%%%%%%%%%%%%%%%%%%%%%%%%%%%%%%%%%%%%%%%%%%%%%%%%%%%%%%%%%%%%

\pgfplotsset{compat=1.16}

\usepackage{dsfont}

\theoI{2}

\begin{document}
\punkteIV{2.1}{2.2}{2.3}{2.4}

\section*{Aufgabe 2.1}

\newpage
\section{Aufgabe 2.2}


\newpage
\section*{Aufgabe 2.3}

a)

Der Geschwindigkeitsvektor ergibt sich aus der Ableitung des Positionsvektors:

$$
\vec{x}(t) = \begin{pmatrix}
R cos(\omega t) \\
R sin(\omega t) \\
v_0 t
\end{pmatrix}
$$

$$
\vec{v	}(t) = \dot{vec{x}}(t) = \begin{pmatrix}
- R \omega sin(\omega t) \\
R \omega cos(\omega t) \\
v_0
\end{pmatrix}
$$

Die Bogenlänge berechnet sich wie folgt:

$$
s(t) =  \int_{t_0}^{t} d t' | \vec{v}(t') | = \int_{t_0}^{t} d t' \sqrt{\vec{v}(t')^{2}} = \int_{0}^{t} d t' \sqrt{R^{2}\omega^{2} ((-sin(\omega t'))^{2} + cos(\omega t')^{2} + v_0^{2}} = \sqrt{R^{2} \omega^{2} + v_0^{2}} t
$$

Man beachte hierbei, dass die Integrationskonstante tatsächlich $0$ ist, da $s(t_0 = 0) = 0$ angenommen wurde.

Die Formel kann umgestellt werden zu:

$$
t(s) = \frac{s}{\sqrt{R^{2} \omega^{2} + v_0^{2}}}
$$

Der Tangentenvektor entspricht dem auf $1$ normierten Geschwindigkeitsvektor:

$$
\vec{T}(s) = \frac{\vec{v}(t(s))}{|\vec{v}(t(s))|} = \frac{1}{\sqrt{R^{2} \omega^{2} + v_0^{2}}} \begin{pmatrix}
- R \omega sin(\frac{\omega s}{\sqrt{R^{2} \omega^{2} + v_0^{2}}}) \\
R \omega cos(\frac{\omega s}{\sqrt{R^{2} \omega^{2} + v_0^{2}}}) \\
v_0
\end{pmatrix}
$$

Für den Krümmungsradius gilt:

$$
\rho = \frac{1}{|\frac{d\vec{T}}{ds}|} = \frac{1}{|\begin{pmatrix}
\frac{- R \omega^{2}}{\sqrt{R^{2} \omega^{2} + v_0^{2}}}cos(\frac{\omega s}{\sqrt{R^{2} \omega^{2} + v_0^{2}}}) \\
\frac{- R \omega^{2}}{\sqrt{R^{2} \omega^{2} + v_0^{2}}}sin(\frac{\omega s}{\sqrt{R^{2} \omega^{2} + v_0^{2}}}) \\
0
\end{pmatrix} |} = \frac{1}{\frac{- R \omega^{2}}{\sqrt{R^{2} \omega^{2} + v_0^{2}}}}
$$

Der Normalenvektor ergibt sich wie folgt:

$$
\vec{N}(s) = \rho \frac{d\vec{T}}{ds} =\begin{pmatrix}
-cos(\frac{\omega s}{\sqrt{R^{2} \omega^{2} + v_0^{2}}}) \\
-sin(\frac{\omega s}{\sqrt{R^{2} \omega^{2} + v_0^{2}}}) \\
0
\end{pmatrix} 
$$

Der Binormalenvektor berechnet sich wie folgt:

\begin{align}
\vec{B}(s) &= \vec{T}(s) \times \vec{N}(s) \\&= \frac{1}{\sqrt{R^{2} \omega^{2} + v_0^{2}}} \begin{pmatrix}
R \omega cos(\omega t(s)) 0 - v_0 (-sin(\omega t(s))) \\
v_0 -cos(\omega t(s)) - 0 (-R \omega sin(\omega t(s)) \\
-R \omega sin(\omega t(s) (-sin(\omega t(s)))  - R \omega \cos(\omega t(s)) (-cos(\omega t(s)))
\end{pmatrix} \\ &= \frac{1}{\sqrt{R^{2} \omega^{2} + v_0^{2}}} \begin{pmatrix}
v_0 (-sin(\omega \frac{s}{\sqrt{R^{2} \omega^{2} + v_0^{2}}})) \\
v_0 (-cos(\omega \frac{s}{\sqrt{R^{2} \omega^{2} + v_0^{2}}})) \\
R \omega
\end{pmatrix}
\end{align}
b)

Aufgrund der alternativen Definition des Vektorprodukts muss gelten, dass $\vec{B}(s) \perp \vec{T}(s)$ und $\vec{B}(s) \perp \vec{N}(s)$. Dass $\vec{B}(s) \perp \vec{N}(s)$ gilt, kann dadurch gezeigt werden, dass das Skalarprodukt $0$ ist:

\begin{align}
\vec{T}(s) \cdot \vec{N}(s) &= \frac{1}{\sqrt{R^{2} \omega^{2} + v_0^{2}}}
((- R \omega sin(\frac{\omega s}{\sqrt{R^{2} \omega^{2} + v_0^{2}}})) (-cos(\frac{\omega s}{\sqrt{R^{2} \omega^{2} + v_0^{2}}})) \\&+ (R \omega cos(\frac{\omega s}{\sqrt{R^{2} \omega^{2} + v_0^2}})) (-sin(\frac{\omega s}{\sqrt{R^{2} \omega^{2}
+ v_0^{2}}})) + 0 v_0 = 0
\end{align}




\newpage
\section{Aufgabe 2.4}











\end{document}